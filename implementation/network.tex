%=========================================================================
% (c) 2011, 2012 Josef Lusticky

\section{Network communication}
The client and the developed interface uses no IP version specific code,
therefore a communication over IPv4 should be also possible.
This was due to a missing solution for
communicating over IPv4 on IEEE~802.15.4 link layer not tested though.

The packet loss problem was described in section~\ref{sec:design-network}.
However, packet loss is not a matter for NTP if using either broadcast or unicast mode.
In broadcast mode, a lost server packet causes no setting or adjusting time of client's
local clock.
The client simply waits without disruption for next NTP broadcast message.
If the client needs to figure out it's local clock offset at the moment,
it can simply query a server using the NTP unicast mode.
% 2
Upon sending the packet, the NTP client process sets the event timer and yields
using the {\it{PROCESS\_YIELD}} statement, so no active waiting
causes blocking the whole system.
The, the process receives an event in response to either an incoming NTP packet
or the etimer expiration.
If any other running application wants the NTP client to query the server,
it can send the {\it{PROCESS\_EVENT\_MSG}} event to the NTP process.
However, no event is sent to that application when the server's response arrives.

The remote NTP server can be specified in the Makefile or
by using the {\it{REMOTE\_HOST}} define macro.
If no remote host is specified, the NTP client assumes that only the NTP broadcast communication mode will be used.
The broadcast mode is intended particularly for energy or memory constrained clients
or for a huge number of NTP clients and a single NTP server
in a network with small propagation delay.
The link layer MAC address and the NTP poll exponent $\tau$
can be configured in the Makefile as well.


%design
%Dynamic increasing or decreasing the client's poll interval in response to
%Kiss-o'-Death packets, described in section~\ref{sec:ntp-network}, is not implemented.
%The configuration instead assumes, that an exhausted NTP server rather drops the incoming
%client's packet than sending the response with KoD code.
