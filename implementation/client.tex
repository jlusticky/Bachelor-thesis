%=========================================================================
% (c) 2011, 2012 Josef Lusticky <xlusti00@stud.fit.vutbr.cz>

\section{NTP client implementation}
Structures representing NTP message were borrowed from OpenNTPD NTP Unix daemon.
%They are not using the GCC extension for representing a bit field.

Packet sanity tests~\cite{ntp-arch}.

A client sends messages to each server with a poll interval of $2^{\tau}$
seconds, as determined by the poll exponent $\tau$ (tau).
In NTPv4, $\tau$ ranges from 4 (16 s) to 17 (36 h).


% filling the packet
Precision express strictly speaking elapsed time to read the system clock from userland~\cite{ntp-arch}.
However most implementation supply clock precision.
%Dragonfly BSD:
%wmsg.precision = -6;
%Chrony, NTP.org - getting resolution, gettimeofday, clock_getres
\begin{lstlisting}
// set clock precision - convert Hz to log2 - borrowed from OpenNTPD
int b = CLOCK_SECOND; // CLOCK_SECOND * OCR2A
int a;
for (a = 0; b > 1; a--, b >>= 1)
  {}
msg.precision = a;
\end{lstlisting}
This will work for clock precision greater or equal 1s, i.e. CLOCK\_SECOND must be greater or equal 1.
%refer to CLOCK\_SECOND - is always greater or equal 1

The clock are set if the time difference is greater than XX seconds. %! TODO
\begin{lstlisting}
if (labs((signed long) (ts.sec - tmpts.sec)) > 2)
{
  clock_set_time(ts.sec);
}
\end{lstlisting}
Even if {\it{tmpts.sec}} value is greater than {\it{ts.sec}} value,
subtracting and casting to signed type gives correct (negative) result~\cite{c99}.
Assuming 32-bit data types this will work until 2038 when wrap around can occur due to difference
between {\it{ts.sec}} and {\it{tmpts.sec}} greater than $2^{31}$-1 (2~147~483~647).
But as NTP Era 0 ends 2036 the NTP client code must be changed in the future anyway.

%! TODO

%Adjusting time
%1/128/32 = 0.000244141
%0.000244141x32x127+0.000244141x31 == smallest possible adjustment == 244us

%\section{NTP values and convertions}
Unlike the RFC 5905~\cite{rfc5905} shows, there are no 64 bit values. %! RFC - A.4. Kernel System Clock Interface
No floating point numbers - library.
Division of unsigned integer number by 2 is automatically translated by compiler to logical right shift,
making it fast operation.

Converting between NTP and local timestamps requires floating point operations or 64-bit numbers.
According to output from avr-size tool, using 64-int number for conversion
uses 4~330 bytes more in program section of resulted binary file 
and floating point operation takes 3~474 bytes more
than algorithm developed?

In case of floating point operations, the libgcc is used.
\url{http://gcc.gnu.org/onlinedocs/gccint/Libgcc.html}

%
% NEGATIVE result for the first time
In some scenarios where the initial frequency offset of the client is
  relatively large and the actual propagation time small, it is
   possible for the delay computation to become negative.  For instance,
   if the frequency difference is 100 ppm and the interval T4-T1 is 64
   s, the apparent delay is -6.4 ms.  Since negative values are
   misleading in subsequent computations, the value of delta should be
   clamped not less than s.rho, where s.rho is the system precision
   described in Section 11.1, expressed in seconds~\cite{rfc5905}.
%
