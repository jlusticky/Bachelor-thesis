%=========================================================================
% (c) 2011, 2012 Josef Lusticky

\section{Network communication}
% 1 - see design/network.tex
...uvod do packet lost + ref
However, packet loss is not a matter for NTP if using either broadcast or unicast mode.
In broadcast mode, lost server packet causes no setting or adjusting time of client's
local clock.
The client simply waits without disruption for next NTP broadcast message.
If client needs to figure out it's local clock offset at the moment,
it can simply query a server using NTP unicast mode.

% 2
... no waiting until packet received
This %was %!TODO
implemented using {\it{PROCESS\_YIELD}} statement.

NTP server can be specified in Makefile or
during compilate time using {\it{REMOTE\_HOST}} define.
%! TODO
TODO: If no host is specified,
NTP client assumes NTP broadcast communication mode.


%The uIP packet buffer is accessed through
%the uip\_buf array and is used to hold incoming and outgoing packets.
%The device driver should place incoming data into this buffer.
%When sending data, the device driver should read the link
%level headers and the TCP/IP headers from this buffer.
%The size of the link level headers is configured by the UIP\_LLH\_LEN
%define and in case of Ethernet it is 14.

%The application data need not be placed in this buffer, so
%the device driver must read it from the place pointed to by the
%uip\_appdata pointer %as illustrated by the following example:

Routing to Meinberg NTP primary server.
Measurements made using this setup are discussed in chapter~\ref{chap:measurements}.
