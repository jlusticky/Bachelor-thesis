%=========================================================================
% (c) 2011, 2012 Josef Lusticky <xlusti00@stud.fit.vutbr.cz>

\section{Hardware clock interface}
Contiki features a basic clock interface with a simple goal - measuring time.
This interface provides needed calls for timers and its definition is to be found in {\it{/core/sys/clock.h}} file.
Specific implementations of this common interface are located in {\it{/cpu}} directory of Contiki source code.
The interface provides call for initialising CPU's clock system {\it{clock\_init}} that is automatically called during
boot sequence of Contiki.
The goal of the {\it{clock\_init}} call is to set up
appropriate counter registers and interrupt service routines. %! TODO .. as described in chapter..
For AVR CPU this call is implemented as C macro, which evaluates to specific setup code for each
different type of AVR CPU during compilation, and is defined in {\it{cpu/avr/dev/clock-avr.h}} file.
The setup code is not common to all CPUs because of differences among them - e.g. there are usually
only three Timer/Counter modules, but AVR ATmega1284P has four Timer/Counter modules~\cite{avr-datasheet}.

On AVR Raven, 8 bit Timer/Counter~2 clocked from 32~768~Hz watch crystal
is used by Contiki clock interface,
which is in turn used by dependent timers - refer to section~\ref{sec:contiki-timers} for details.
This 32~768~Hz watch crystal is independent of the I/O clock, can be only used
with Timer/Counter~2 and it
enables use of Timer/Counter2 as a Real Time Counter~\ref{avr-datasheet}.
%Unlike I/O clock used for clocking other Timers/Counters,
%this asynchronous crystal is also active in power-save mode~\ref{avr-datasheet}.
If Timer/Counter2 is enabled, it will keep running during sleep. The device can wake up from
either Timer Overflow or Output Compare event from Timer/Counter2 if the corresponding
Timer/Counter2 interrupt enable bits are set in TIMSK2, and the Global Interrupt Enable bit in
SREG is set.
If Timer/Counter2 is not running, Power-down mode is recommended instead of Power-save
mode.
The Timer/Counter2 can be clocked both synchronously and asynchronously in Power-save
mode. If the Timer/Counter2 is not using the asynchronous clock, the Timer/Counter Oscillator is
stopped during sleep. If the Timer/Counter2 is not using the synchronous clock, the clock source
is stopped during sleep. Note that even if the synchronous clock is running in Power-save, this
clock is only available for the Timer/Counter2.


Timer/Counter~2 is used in Clear Timer on Compare Match (CTC) mode by Contiki.
In this mode the counter register (TCNT2) is cleared to zero when the counter
value matches the value in output compare register (OCR2A).
The OCR2A register defines the top value for the counter, hence also its resolution.

% PRESCALER
For Timer/Counter2, the possible prescaled selections are: clk T2S /8, clk T2S /32, clk T2S /64,
clkT2S/128, clkT2S/256, and clkT2S/1024. Additionally, clkT2S as well as 0 (stop) may be selected.

% write to OCR2A
If the interrupt is enabled, the interrupt handler routine can be used for updating
the TOP value. However, changing TOP to a value close to BOTTOM when the counter is run-
ning with none or a low prescaler value must be done with care since the CTC mode does not
have the double buffering feature. If the new value written to OCR2A is lower than the current
value of TCNT2, the counter will miss the compare match.

CTC

$f = $

%At least one of those is always 16 bit wide
%This extra module on AVR ATmega1284P is used for
% three vs. 3

%clock\_seconds
%CLOCK\_SECOND
This is however enough for implementing a reasonable time interface and using it for NTP client later.

% ntp interface extending the clock library, similar to posix calls

%!!AVR

Adjusting time - COMPARE\_REGISTER = 31 => 128Hz => 1s = 1s
FREQ = 32768/8 / 32
COMPARE\_REGISTER = 30 => ca132.129Hz => 1s = ca1.032258s
FREQ = 32768/8 / 31
COMPARE\_REGISTER = 32 => 124.12per => 1s = 0.96p
FREQ = 32768/8 / 33

=> fastest adjust is 0,03s / s


Each TCNT2 increment is $\frac{1}{128 \times 32} \doteq 0,000244$ s
2,44ms minimum clock slew
This is also minimal possible clock adjustment.


Adjustments will influence the timers.
Applications requiring uninfluenced timers
are therefore advised to use rtimers, described in section~\ref{sec:contiki-timers},
because they use separate hardware clock
unaffected by NTP client.
