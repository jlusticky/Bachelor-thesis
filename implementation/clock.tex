%=========================================================================
% (c) 2011, 2012 Josef Lusticky

\section{Clock library extension}\label{sec:implementation-clock}
Section~\ref{sec:analysis-clock-interface} described, that
the {\it{clock\_init}} call evaluates to a specific setup code
for each different AVR CPU during the compilation.
Other AVR CPUs can be using a different hardware clock.
However, the time interface is common for all AVR CPUs.
Therefore, new general names for each part of the hardware clock
must be defined in the {\it{clock\_init}} call.

The compare register {\it{OCR2A}} is defined by macro as {\it{CLOCK\_COMPARE\_REGISTER}},
the counter register {\it{TCNT2}} is defined as {\it{CLOCK\_COUNTER\_REGISTER}},
the default value of the clock compare register, when no time adjustments are in progress,
is defined as {\it{CLOCK\_COMPARE\_DEFAULT\_VALUE}}
and {\it{CLOCK\_CTC\_MODE}} is defined as 1, because the hardware clock is used in CTC mode,
as described in section~\ref{sec:analysis-hwclock}.
So in fact, the code presented in the previous section is not a pseudocode.
Such an extension also makes porting the code to other platforms simple.

The {\it{clock\_adjust\_time}} function uses the new {\it{adjcompare}} variable,
as described in the previous section.
The data type of this variable was chosen to be of the signed 16-bit type.
The limit imposed on time adjustments is therefore $2^{15}$ counter register increments
for slowing down the clock and $2^{15}-1$ for speeding up the clock.
This equals to $2^{15}~\times~0.000244140625 = 8$~seconds
and $(2^{15} - 1)~\times~0.000244140625 \doteq 7.999756$~seconds, respectively.
A wider data type of the {\it{adjcompare}} variable would cause greater scope,
but also an additional memory overhead.
Since it has to be possible to adjust the time at least within one second,
a smaller data width of {\it{adjcompare}} would cause too small scope.
The {\it{volatile}} modifier must be used in conjunction with this variable,
because the variable may be updated in the interrupt service routine.
Listing~\ref{lst:implementation-isr} shows the variable definition
and its use in the interrupt service routine for adjusting the time.
\begin{lstlisting}[caption={Pseudocode of adjustments in interrupt service routine},label={lst:implementation-isr}]
volatile int16_t adjcompare;

ISR(AVR_OUTPUT_COMPARE_INT)
{
  ...
  if (adjcompare == 0) {  // if not adjusting
    CLOCK_COMPARE_REGISTER = CLOCK_COMPARE_DEFAULT_VALUE;
  } else if (adjcompare > 0) {  // if slowing down
    adjcompare--;
    CLOCK_COMPARE_REGISTER = CLOCK_COMPARE_DEFAULT_VALUE + 1;
  } else {  // if speeding up
    adjcompare++;
    CLOCK_COMPARE_REGISTER = CLOCK_COMPARE_DEFAULT_VALUE - 1;
  }
  ...
}
\end{lstlisting}
