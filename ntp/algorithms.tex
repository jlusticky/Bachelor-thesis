%=========================================================================
% (c) 2011, 2012 Josef Lusticky <xlusti00@stud.fit.vutbr.cz>

\section{Algorithms}\label{sec:ntp-algorithms}
Because of network latency the received Transmit Timestamp will never be exactly
corresponding to the current time.
One of the main goals of NTP is to deal with the network latency~\cite{ntp-overview}.

As described in section~\ref{sec:ntp-network},
there are following 64-bit long timestamps in NTP packet: Origin, Receive and Transmit Timestamp.
Upon NTP packet arrival, the client determines another timestamp called
Destination Timestamp~\cite{rfc5905}.
This timestamp is represented as T4 on figure~\ref{fig:ntp-client-server}
and is not part of NTP packet structure.

Using these four timestamps, NTP client can compute
the local clock offset which is given by $\theta = \frac{1}{2}[(T_2 - T_1) + (T_3 - T_4)]$,
where $T_1$ is the time of the request packet transmission (Origin Timestamp),
$T_2$ is the time of the request packet reception (Receive Timestamp),
$T_3$ is the time of the response packet transmission (Transmit Timestamp) and
$T_4$ is the time of the response packet reception (Destination Timestamp)~\cite{ntp-algor,rfc5905}.
The implicit assumption in the above is that one-way delay is
statistically half of round-trip delay~\cite{rfc5905},
which is given by $\delta = (T_4 - T_1) - (T_3 - T_2)$.

When computing result from more servers the intersection algorithms is used
for selecting the possible most exact timestamp received from various servers~\cite{rfc5905}.
Intersection algorithm is derived from Murzollo algorithm but the basic
computation remains the same~\cite{ntp-history}.
The resulting exact timestamp does not have to be the same
as one of those servers provided.
First of all a set of bad and good servers must be made.
Bad servers are called Falsetickers and good are called Truechimers~\cite{rfc5905}.
The division to these sets is based on their response.
As one can assume for sensible result there must be more Truechimers than Falsetickers~\cite{rfc5905}.

Intersection algorithm computes with estimates converted to intervals.
Figure~\ref{fig:ntp-intersection} shows the computation for the following example:
If we have the estimates $10 \pm 2$, $12 \pm 1$ and $11 \pm 1$
then these intervals are $<8; 12>$, $<11; 13>$ and $<10; 12>$ which
intersect to form $<11; 12>$ or $11.5 \pm 0.5$ as consistent with all three values.
The arithmetic mean is used as a value of result.
When querying servers again, the algorithm repeats but the new result computation
also depends on the previous result~\cite{rfc5905}.
This eliminates possible jitter which can be caused by repeatedly querying the servers
and getting slightly different answers from them.

\begin{figure}
	\centering
	\includegraphics[width=13cm,keepaspectratio]{fig/Marzullo_example-1.jpg}
	\caption{Intersection algorithm by D. Exb}
	\label{fig:ntp-intersection}
	\bigskip
\end{figure}

%Since the clients complying with a subset of NTP, called
%the Simple Network Time Protocol (SNTPv4) [RFC4330], do not need to
%implement the mitigation algorithms ... ~\cite{rfc5905}.
