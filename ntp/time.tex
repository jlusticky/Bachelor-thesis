%=========================================================================
% (c) 2011, 2012 Josef Lusticky <xlusti00@stud.fit.vutbr.cz>

\section{Time and timescales}\label{sec:ntp-time}
NTP always uses the Coordinated Universal Time (UTC)~\cite{rfc5905}.
UTC is maintained by the International Bureau of Weights and Measures in Paris, France.
It is the time scale that forms the basis for the coordinated dissemination
of standard frequencies and time signals~\cite{bipm-utc}.
The time specified by UTC is the same for all timezones.
Its calculation is the same as with Greenwich Mean Time (GMT) however
the daylight savings are not accounted.

The UTC scale is adjusted by the insertion of leap seconds to ensure approximate
agreement with the time derived from the rotation of the Earth~\cite{bipm-utc}.
The atomic clocks, on which is UTC based, are so precise that
they do not match the rotation of the Earth
which periodically speeds up and slows down due to the action
of the tides and changes within the Earth's core~\cite{cnn-earth}.
The goal of a leap second is to catch up UTC with these changes.
The leap second is inserted on the advice of the International Earth Rotation and Reference Systems Service~\cite{bipm-utc}.
NTP is well designed for leap second occurance -
there is Leap Indicator field
in the structure of NTP packet and there are also fields intended for
information about leap second in structures NTP algorithm uses~\cite{rfc5905}.

The system time is represented by the system clock maintained by
the hardware and operating system.
The goal of the NTP algorithms is to minimize
both the time difference and frequency difference between UTC and the system clock.
When these differences have been reduced below nominal
tolerances, the system clock is said to be synchronised to UTC~\cite{rfc5905}.
It has never been a goal of NTP to take care of local time,
it is up to operating system to provide user the correct local time~\cite{ntp-overview}.

The NTP and POSIX timescales are based on the UTC timescale,
but not always coincident with it~\cite{ntp-leap}.
Both timescales reckon in seconds since the prime epoch,
but the origin of the NTP timescale, the NTP prime epoch, is 00:00:00 UTC on 1 January 1900,
while the prime epoch of the POSIX timescale is 00:00:00 UTC on 1 January 1970~\cite{ntp-leap}.
So upon the first tick of POSIX on 1 January 1970 the NTP clock read 2~208~988~800,
representing the number of seconds since the NTP prime epoch.
As already mentioned, some of interesting dates
can be found in appendix~\ref{app:dates}.
