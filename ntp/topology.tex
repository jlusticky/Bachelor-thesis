%=========================================================================
% (c) 2011, 2012 Josef Lusticky <xlusti00@stud.fit.vutbr.cz>

\section{Topology and hierarchy}\label{sec:ntp-topology}
NTP uses two different communication modes:
one to one, referred as unicast, and one to many, referred broadcast~\cite{rfc5905}.
In unicast communication mode, NTP client sends request and NTP server sends response.
In broadcast communication mode, the client sends no request
and waits for a broadcast message from one or more servers~\cite{rfc5905}.

NTP servers are rated with stratum (plural form strata) number which represents their position
in an NTP hierarchy and their possible accuracy~\cite{rfc5905}.
Primary (stratum 1) servers synchronise to the reference clock directly traceable to UTC via
radio, satellite or modem.
The stratum 2 servers synchronise to stratum 1
servers via hierarchical subnet.
The stratum 3 servers synchronise to stratum 2 servers, and so on.
The maximum stratum is 15, number 16 means unsynchronised server
and higher numbers (up to 255) are reserved~\cite{rfc5905}.
Synchronisation between servers in the same stratum level is also possible.
Figure~\ref{fig:ntp-hierarchy} shows a brief hierarchy of NTP.
\begin{figure}
  \centering
  %\input{./xfig/test.pstex_t}
  \includegraphics[width=9cm,keepaspectratio]{fig/Network_Time_Protocol_servers_and_clients.pdf}
  \caption{Topology and hierarchy of NTP by B. Esham}
  \label{fig:ntp-hierarchy}
  \bigskip
\end{figure}
