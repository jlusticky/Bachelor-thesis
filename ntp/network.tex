%=========================================================================
% (c) 2011, 2012 Josef Lusticky <xlusti00@stud.fit.vutbr.cz>

\section{Network and timestamps}\label{sec:ntp-network}
Network specification of NTP defines that
the protocol uses the User Datagram Protocol (UDP) on port number 123~\cite{ianna-ports,rfc5905}.
Reliable message delivery such as TCP can actually make the delivered of
NTP packet less reliable since retries
would increase the delay value and other errors~\cite{rfc5905}.
This is mostly due to overhead of communication with TCP on transport layer.

NTP manipulates with the time through timestamps - a record of time.
NTP timestamp has two fields: the seconds field expressing the number of seconds
and the fraction field expressing fraction of a second~\cite{rfc5905}.
All NTP time values are represented in twos-complement format, with
bits numbered in big-endian fashion from zero starting at the left, or high-order, position~\cite{rfc5905}. 
There are two formats of timestamp in NTP packet structure:
long 64-bit and short 32-bit as shown on figure~\ref{fig:ntp-timestamps}.
The 64-bit long timestamp used by NTP consists of a 32-bit unsigned seconds
field spanning $2^{32}$ seconds (approx. 136 years from 1900 to 2036) and a 32-bit fraction field resolving
$2^{-32}$ seconds (approx. 232 picoseconds)~\cite{rfc5905}.
The short 32-bit timestamp includes a 16-bit unsigned seconds field
and 16-bit fraction field.

\begin{figure}
	\centering
	\includegraphics[width=13cm,keepaspectratio]{fig/ntp-timestamps.pdf}
	\caption{Time formats used in NTP packet}
	\label{fig:ntp-timestamps}
	\bigskip
\end{figure}

Besides these two, there is one more NTP timestamp format - 128-bit NTP Date format.
It includes a 64-bit signed seconds field and 64-bit fraction field.
For convenience in mapping between formats,
the seconds field is divided into a 32-bit Era Number field
and a 32-bit Era Offset field.
This 128-bit NTP Date format is however not transmitted over the network
and is only used where sufficient storage and word size is available~\cite{rfc5905}.
There is practically no need of knowing about this format for embedded systems
at least until year 2036, when the Era Number will be incremented from zero to one.
But strictly speaking an NTP timestamp is a truncated NTP Date format~\cite{rfc5905}.
Refer to appendix~\ref{app:dates} for a short list of some dates
requiring usage of NTP Date format.

\begin{figure}
	\centering
	\includegraphics[width=9cm,keepaspectratio]{fig/ntp-packet.pdf}
	\caption{Basic NTP packet structure by D. Mills}
	\label{fig:ntp-packet}
	\bigskip
\end{figure}

Standard NTP packet structure without extension fields and
Autokey security protocol is shown on figure~\ref{fig:ntp-packet}.
This structure is 48 bytes long and contains following fields:
\begin{itemize}
\item
Leap Indicator is 2-bit integer warning of an impending leap
second to be inserted or deleted in the last minute of the current
month~\cite{rfc5905}.
\item
Version Number is 3-bit integer representing the NTP
version number, currently~4.
\item
Mode is 3-bit integer representing protocol mode.
This is the only field distinguish between server and client in NTP.
In client-server communication model, the client sets this field to value 3 (client) in the request,
and the server sets it to value 4 (server) in the reply.
In broadcast communication model, the client sends no request
and waits for a broadcast message from one or more servers.
The server sets this field to value 5 (broadcast).
Other modes are not used by SNTP servers and clients~\cite{rfc4330}.
\item
Stratum is 8-bit integer representing the stratum as described in section~\ref{sec:ntp-topology}.
If the Stratum field is 0, which implies unspecified or invalid, the
Reference Identifier field can be used to convey messages useful for
status reporting and access control.
These are called Kiss-o'-Death (KoD)
packets and the ASCII messages they convey are called kiss codes~\cite{rfc5905}.
\item
Poll is 8-bit signed integer representing the maximum interval between
successive messages, in log2 seconds.
Suggested default limits for minimum and maximum poll intervals are 6 and 10, respectively~\cite{rfc5905}.
\item
Precision is 8-bit signed integer representing the precision of the
system clock, in log2 seconds.
For instance, a value of -20
corresponds to a precision of about one microsecond ($2^{-20}$~s)~\cite{rfc5905}.
\item
Root Delay is total round-trip delay to the reference clock, in NTP short 32-bit format~\cite{rfc5905}.
\item
Root Dispersion is total dispersion to the reference clock, in NTP short 32-bit format~\cite{rfc5905}.
\item
Reference Identifier is 32-bit code identifying the particular server used for synchronisation
or reference clock.
For packet stratum 0, this is four-character ASCII string called kiss code.
Kiss codes are particularly used by server to tell the client to stop sending packets or
to increase its polling interval.
For stratum 1, this is a four-octet, left-justified, zero-padded ASCII
string assigned to the reference clock (e.g. "GPS" when synchronising against Global Position System clock).
Above stratum 1, this is the reference identifier of the server used for synchronisation
and can be used by client together with stratum field to detect loops in NTP hierarchy.
If communicating over IPv4, the identifier is the IPv4 address.
If communicating over IPv6, it is the first four octets of the MD5 hash of the IPv6 address~\cite{rfc5905}
\item
%!TODO
TODO
\end{itemize}

Because Root dispersion and Root Delay fields do not need so big scope and precision,
the short 32-bit format is used for them.

% RFC 1769
The Originate Timestamp field is copied
   unchanged from the Transmit Timestamp field of the request. It is
   important that this field be copied intact, as a NTP client uses it
   to check for replays.
  ?? In broadcast mode, this field is set to the
   time of day when the message is sent.
%

%TODO - TIMESTAMPS
There exists approx. 232-picosecond interval, henceforth ignored, every 136 years when
the 64-bit field will be zero, which by convention is interpreted as an invalid timestamp.
%TODO
