%=========================================================================
% (c) 2011, 2012 Josef Lusticky <xlusti00@stud.fit.vutbr.cz>

\section{uIP}\label{sec:contiki-uip}
The TCP/IP protocol family is often used for communication over the Internet as well as local networks.
uIP (micro IP) is a complete TCP/IP communicating stack developed by Adam Dunkels for memory constrained systems such as embedded systems.


uIP provides two different APIs to programmers: a BSD sockets-like API based on Protothreads called Protosockets
and raw event-driven API.

Protosockets are based on Protothreads putting the same limitation on them - such as 
\!
Protosockets only work with Transport Control Protocol (TCP) connections \cite{contiki-docs}.
Since NTP uses User Datagram Protocol (UDP) Protosockets will not be
discussed in this thesis anymore.

In order to reduce amount of memory requirements and code size the
uIP implementation uses an event-based Application programming interface (API).
The application is invoked in response to certain events.
It is up to application that is receiving events from uIP to handle all
work with data to be transmitted. E.g. if the data is lost in the network,
the application will be invoked with the uip\_rexmit event being set.
The application will then have to resend the data.
This approach is based on the fact that it should be an easy work for application
to rebuild the same data.

\!This way uIP does not have to have its own buffer and thus
does not need


This is fundamentally different from the most common TCP/IP API, the BSD sockets API \cite{thesis-programming}.
\!!


Before uIP ...TCP/IP heavyweight...

uIP contains only the absolute minimum of required features to fulfill the protocol standard.
It can only handle a single network interface and contains the IP, ICMP, UDP and TCP protocols.
Despite being so small uIP is not only RFC compliant but also IPv6 Ready Phase 1 certified.
uIP is written in the C programming language and it is fully integrated to Contiki operating system.
