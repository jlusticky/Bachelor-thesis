%=========================================================================
% (c) 2011, 2012 Josef Lusticky

\section{uIP}\label{sec:contiki-uip}
The TCP/IP protocol suite is often used for communication over the Internet as well as local networks.
uIP (micro IP) is a complete TCP/IP communication stack developed by Adam Dunkels
for memory constrained systems such as embedded systems.

Before uIP, the TCP/IP architecture was considered to be heavyweight
because of its perceived need for processing power and memory.
The IP protocol was seen as too large to fit into the constrained environment -
existing implementations of the IP protocol family for general purpose computers would need hundreds
of kilobytes, whereas a typical constrained system has only a few tens of kilobytes of memory~\cite{interconnecting}.
Several non-IP stacks were developed for this reason.

In the early 2000's, however, this view was challenged by lightweight implementations of the IP
protocol family for smart objects such as the uIP stack~\cite{interconnecting}.
uIP showed that the IP architecture would fit nicely into the typical constrained system
without removing any of the essential IP mechanisms.
Note that these resources, which are considered constrained today, are fairly close to the
resources of general purpose computers that were available when IP was designed~\cite{interconnecting}.
The uIP stack has become widely used in networked embedded systems
since its initial release~\cite{interconnecting, thesis-programming}.

uIP provides two different application programming interfaces to programmers:
a BSD sockets-like API called Protosockets and a raw event-driven API.
Protosockets are based on Protothreads, which puts the same limitations on them - such as
no way to store the local variables and an impossibility to use C {\it switch} statements.
Protosockets only work with TCP connections~\cite{contiki-docs}.
Because NTP uses the User Datagram Protocol, Protosockets will not be further
discussed in this thesis. For more information about Protosockets
please refer to the Contiki documentation~\cite{contiki-docs}.

uIP contains only the absolute minimum of required features to fulfill the protocol standard.
It can handle only a single network interface and contains the IP, ICMP, UDP and TCP protocols~\cite{contiki-docs}.
In order to reduce memory requirements and code size,
the uIP implementation uses an event-based API, which is fundamentally different
from the most common TCP/IP API, the BSD sockets API, present on Unix-like systems
and defined by the POSIX standard~\cite{thesis-programming,posix}.
An application is invoked in response to certain events and
it is up to the application that is receives the events from uIP to handle all the
work with data to be transmitted - e.g. if the data packet is lost in the network,
the application will be invoked and it has to resend the packet.
This approach is based on the fact that it should be simple for the application
to rebuild the same data.
This way, the uIP stack does not need to use explicit dynamic memory allocation.
Instead, it uses a single global buffer for holding the packets and it has a fixed
table for holding the connection state.
The global packet buffer is large enough to contain one packet of the maximum size~\cite{contiki-docs}.

When a packet arrives from the network, the device driver places it in the
global buffer and calls the uIP stack.
If the packet contains data, the uIP stack will notify the corresponding application.
Because the data in the buffer will be overwritten by the next incoming packet,
the application will either have to act immediately on the data or copy the data into
its own buffer for later processing.
The packet buffer will not be overwritten by new packets before the application has processed the data~\cite{contiki-docs}.
Packets arriving when the application is processing the data must be queued
either by the network device or by the device driver.
This means that uIP relies on the hardware when it comes to buffering.
Most single-chip Ethernet controllers have on-chip buffers
that are large enough to contain at least 4 maximum sized Ethernet frames~\cite{contiki-docs}.
This way, uIP does not have to have its own buffer structures and thus requires only a minimal memory amount.
A possible packet loss is a trade-off for minimal memory requirements.
It is not such a big problem for communication using TCP on the transport layer
because of the acknowledgement scheme used in TCP to prevent data loss.
However, data carried on UDP can be irrecoverably lost.

As was expected, measurements show that the uIP implementation provides very low
throughput, particularly when it communicates with a PC host~\cite{thesis-towards}.
However, small systems that uIP is targeting, usually do not produce enough data
to make the performance degradation a serious problem~\cite{thesis-towards}.

Despite being so small, uIP is not only RFC compliant, but also IPv6 Ready Phase 1 certified.
uIP is written in the C programming language and it is fully integrated with the Contiki operating system.
In uIP, there are even some more tricks to shrink the stack
but its complete description is outside the scope of this thesis.
Please refer to the Contiki documentation for more details~\cite{contiki-docs}.
