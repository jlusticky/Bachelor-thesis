%=========================================================================
% (c) 2011, 2012 Josef Lusticky <xlusti00@stud.fit.vutbr.cz>

\section{Rime}\label{sec:contiki-rime}
Typical sensor devices are equipped with 8-bit microcontrollers,
code memory on the order of 100 kilobytes, and less than
20 kilobytes of RAM~\cite{paper-contiki}.
Moore's law predicts that these devices
can be made significantly smaller and less expensive
in the future. While this means that sensor networks can
be deployed to greater extents, it does not necessarily imply
that the resources will be less constrained~\cite{paper-contiki}.

The purpose of Rime is to simplify implementation of
sensor network protocols and facilitate code reuse~\cite{paper-rime}. 
The communication primitives in the Rime stack were choosen
based on what typical sensor network protocols use~\cite{contiki-docs}.
Rime can significantly simplify protocol implementation
with only a small increase in resource requirements.
The code footprint of Rime is less than two kilobytes and the
data memory requirements on the order of tens of bytes~\cite{paper-rime}.
To reduce memory footprint Rime uses a single buffer for
both incoming and outgoing packets similar to uIP. Layers
that need to queue data copy the data to dynamically
allocated queue buffers~\cite{paper-rime}.

Rime is undoubtely an interesting communication stack, but
as NTP is designed to run on top of IP, Rime will not be discussed
in this thesis further.
Please consult the Contiki documentation~\cite{contiki-docs} if you desire
more details about Rime.
