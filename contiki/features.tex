%=========================================================================
% (c) 2011, 2012 Josef Lusticky

\section{Features}
Contiki OS features lightweight stackless threads called Protothreads.
Protothreads are a new concept brought to the embedded world by Contiki,
they are extremely lightweight and compatible with standard C~\cite{paper-protothreads}.
Each Protothread does not require a separate stack which fits them perfectly
for usage in memory constrained embedded systems.
Protothreads are more detailed discussed in section~\ref{sec:contiki-protothreads}.

Contiki also features TCP/IP communication stack called uIP~(micro~IP)
that conforms to Request For Comments memorandums published by the Internet Engineering Task Force.
The uIP provides communication abilities using both IPv4 and IPv6~\cite{contiki-docs}.
Contiki with its uIP stack is IPv6 Ready Phase 1 certified
and therefore has right to use the IPv6~Ready silver logo~\cite{ipv6ready-db}.
Before Contiki's uIP, the embedded world considered IP to be too heavyweight.
That means all earlier IP implementations for general purpose computers
were much bigger than the memory constrained embedded systems could use~\cite{interconnecting}.
The communication stack uIP is closely described in section~\ref{sec:contiki-uip}.

Next to the uIP, Contiki is equipped with other communication stack called Rime.
Rime is a layered communication stack for sensor networks,
with much thinner layers than traditional architectures~\cite{paper-rime}.
Rime is designed to simplify the implementation of communication
protocols on low-power radios.
The communication primitives in the Rime stack were chosen
based on what typical sensor network protocols use -
single-hop unicast, single-hop broadcast or multi-hop~\cite{contiki-docs,paper-rime}.

Beside Protothreads, uIP and Rime,
Contiki contains a very simple, relatively small and easy to use filesystem
called Coffee Filesystem (CFS),
a graphical user interface called Contiki Toolkit (CTK),
Executable Linkable Format (ELF) loader for loading object files into a running Contiki system
and much more.

Operating system Contiki, uIP and Protothreads are used by hundreds of companies in embedded devices in
such diverse systems as car engines, oil boring equipment, satellites, and container security systems~\cite{thesis-programming}.
The software is also used both in academic research
projects and in university project courses on embedded systems throughout the
world.

\begin{figure}
  \centering
  \includegraphics[width=9cm,keepaspectratio]{fig/contiki-vnc.png}
  \caption{Screenshot of running Contiki OS with CTK (source:~\cite{contiki-docs})}
\end{figure}
