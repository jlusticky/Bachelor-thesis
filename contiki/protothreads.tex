%=========================================================================
% (c) 2011, 2012 Josef Lusticky <xlusti00@stud.fit.vutbr.cz>

\section{Protothreads}\label{sec:contiki-protothreads}
Protothreads provide a way to run C functions quasi-paralelly, that is, a C functions work in a way similiar to thread.
In Contiki Protothreads allow process to wait for incoming events. While waiting for an event to occur another function
can be run. The core of this solution is C switch statement used in conjuction with variable (called local continuation)
containing the position where the function was interrupted. Next time function continues from this point.

The advantage of Protothreads over ordinary threads is that a Protothread does not require a separate stack.
In memory constrained systems, the overhead of allocating multiple stacks can consume large amounts of
the available memory. In contrast, each Protothread only requires few bytes for storing the state of execution.

A Protothread is driven by repeated calls to the function in which the Protothread is running. Each time the
function is called, the Protothread will run until it blocks or exits.
Protothreads are implemented using local continuations. A local continuation represents the current state
of execution at a particular place in the program, but does not provide any call history or local variables.

The Protothreads API consists of four basic operations: initialization (PT\_INIT()), execution (PT\_BEGIN()),
conditional blocking (PT\_WAIT\_UNTIL()) and exit (PT\_END()). On top of these, two convenience functions
are built: reversed condition blocking (PT\_WAIT\_WHILE()) and Protothread blocking (PT\_WAIT\_THREAD())
\cite{paper-protothreads}.


Here is an example of delaying text scrolling on an LCD panel using Protothreads. This example is taken from
Adam Dunkel's Protothreads site \cite{adam-protothreads} and slightly modified.
Protothreads can be used for introducing delays inside a function, without using a full threading model.
The following example shows a function writing text to a one-line LCD panel.
If the text is longer than the size of the panel, the text should be scrolling in from the right. 

\begin{lstlisting}
#include "pt.h"
#include "timer.h"

typedef unsigned short lc_t;

struct pt {
  lc_t lc;         /* Local continuation */
}; 

struct state {
  char *text;
  char *scrollptr;
  struct pt pt;
  struct timer timer;
};

static struct state s;
 
PT_THREAD(display_text(struct state *s))
{
  PT_BEGIN(&s->pt);
 
  /* If the text is shorter than the display size, show it right
     away. */
  if(strlen(s->text) <= LCD_SIZE) {
    lcd_display_text(s->text);
  } else {    
    /* If the text is longer than the display, we should scroll in the
       text from the right with a delay of one second per scroll
       step. We do this in a for() loop, where the loop variable is
       the pointer to the first character to be displayed. */
    for(s->scrollptr = s->text;
        strlen(s->scrollptr) > LCD_SIZE;
        ++s->scrollptr) {
      lcd_display_text(s->scrollptr);
 
      /* Wait for one second. */
      timer_set(&s->timer, ONE_SECOND);
      PT_WAIT_UNTIL(&s->pt, timer_expired(&s->timer));
    }
  }
  
  PT_END(&s->pt);
}

int main(void)
{
  PT_INIT(&s);
  s.text = "Long text for LCD display";
  for (;;) {
    display_text(&s);
    /* Here can be another thread run */
  }
  return 0;
}

\end{lstlisting}

The {\it PT\_WAIT\_UNTIL} macro actually causes the function
to return. While the function is waiting for the timer to expire another function can called and run.
When the function is entered again the execution continues with the {\it PT\_WAIT\_UNTIL} macro
which causes the function to check the condition it is waiting for (timer expired). If the condition is
met the function resumes, if not it returns again.

How does it work? The macro {\it PT\_BEGIN} exapands to {\it switch} statement while preprocessing the
code for compilation. The {\it PT\_WAIT\_UNTIL} macro expands to {\it case} and setting the local continuation
to the value, so that next time this function is run, it jumps to this {\it case}.
The structure holding the state is defined outside of the function so its context is not lost when
the function returns. The simplest state structure would hold just the local continuation variable.

Follows the same sample of code after partial simplified preprocessing.

\begin{lstlisting}
#include "pt.h"
#include "timer.h"

typedef unsigned short lc_t;

struct pt {
  lc_t lc;         /* Local continuation */
}; 

struct state {
  char *text;
  char *scrollptr;
  struct pt pt;
  struct timer timer;
};
 
int display_text(struct state *s)       /* Expanded PT_THREAD macro */
{
  switch((&s->pt)->lc) {  case 0:      /* Expanded PT_BEGIN(&s->pt); */
 
  /* If the text is shorter than the display size, show it right
     away. */
  if(strlen(s->text) <= LCD_SIZE) {
    lcd_display_text(s->text);
  } else {    
    /* If the text is longer than the display, we should scroll in the
       text from the right with a delay of one second per scroll
       step. We do this in a for() loop, where the loop variable is
       the pointer to the first character to be displayed. */
    for(s->scrollptr = s->text;
        strlen(s->scrollptr) > LCD_SIZE;
        ++s->scrollptr) {
      lcd_display_text(s->scrollptr);
 
      /* Wait for one second. */
      timer_set(&s->timer, ONE_SECOND);
      (&s->pt)->lc = 31; case 31: if(!(timer_expired(&s->timer))) { return PT_WAITING; }
    }
  }
  
  (&s->pt)->lc = 0; return PT_ENDED; }   /* Expanded PT_END */
  
}

int main(void)
{
  (&s)->lc = 0;                         /* Expanded PT_INIT */
  s.text = "Long text for LCD display";
  for (;;) {
    display_text(&s);
    /* Here can be another thread run */
  }
  return 0;
}

\end{lstlisting}


Since Protothreads are implemented using pure C, library providing Protothreads can be used everywhere C toolchain is available.

But there are some cons to consider. Because protothreads are stackless, a Protothread can only run within a single C function.
There is also no way of storing automatic local variables. And since Protothreads are implemented using C {\it switch} statement, and these can
not be nested, the code that uses Protothreads cannot use {\it switch} statements itself.
Workaround for storing local variables is to prepend them with the {\it static} keyword, which make them being put into data segment
by compiler and thus remembering the value between the function calls.

