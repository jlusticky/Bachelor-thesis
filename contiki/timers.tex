%=========================================================================
% (c) 2011, 2012 Josef Lusticky <xlusti00@stud.fit.vutbr.cz>

\section{Timers}\label{sec:contiki-timers}
As the Contiki kernel kernel does not provide support for timed events,
an application that wants to use timers needs to explicitly use the timer library.
The timer library provides functions for setting, resetting and restarting timers,
and for checking if a timer has expired~\cite{contiki-docs}.
An application must manually check if its timers have expired - this is not done automatically~\cite{contiki-docs}.


Contiki has one clock module and a set of timer modules: timer, stimer, ctimer, etimer, and rtimer~\cite{contiki-wiki-timers}.


The timer library is not able to post events when a timer expires. The Event timers should be used for this purpose.
The timer library uses the Clock library to measure time. Intervals should be specified in the format used by the clock library.

The clock library is the interface between Contiki and the platform specific clock functionality.
The clock library performs a single function: measuring time. Additionally, the clock library provides a macro, CLOCK\_SECOND, which corresponds to one second of system time.

The clock library need in many cases not be used directly. Rather, the timer library or the event timers should be used.
