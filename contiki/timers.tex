%=========================================================================
% (c) 2011, 2012 Josef Lusticky

\section{Timers}\label{sec:contiki-timers}
The Contiki kernel does not provide support for timed events,
instead an application that wants to use timers needs to explicitly use a timer library.
The timer library provides functions for setting, resetting and restarting timers,
and for checking if a timer has expired.
An application must manually check if its timers have expired - this is not done automatically~\cite{contiki-docs}.

Contiki has one clock library and a set of timer libraries: timer, stimer, ctimer, etimer, and rtimer~\cite{contiki-wiki-timers}.
The clock library provides functionality to handle the system time and also to block the CPU for short time periods.
It is the interface between Contiki and the platform specific clock functionality~\cite{contiki-docs}.
The timer libraries are implemented with the functionality of the clock library as a base~\cite{contiki-wiki-timers}.

The timer and stimer libraries provide the simplest form of timers and are used to check if a time period has passed.
The difference between these two is the resolution of time -
timers use system clock ticks, whose value is incremented when an interrupt from the hardware clock occurs,
while stimers use seconds to offer much longer time periods~\cite{contiki-wiki-timers}.
The value representing seconds is also incremented in the interrupt service routine (ISR),
but only when enough clock ticks since last increment occurred.
The number of clock ticks within one second is represented by the
{\it{CLOCK\_SECOND}} macro provided by the clock library.
That means there are {\it{CLOCK\_SECOND}} interrupts from the hardware clock per second.
The usage of the timer library and {\it{CLOCK\_SECOND}} macro is shown in appendix~\ref{app:protothreads}.

The simplest timer and stimer libraries are not able to post an event when a timer expires.
Event timers should be used for this purpose.
Event timers (etimer library) provide a way to generate timed events.
An event timer will post an event to the process that set the timer when the
event timer expires~\cite{contiki-docs}.
The etimer library is implemented as a Contiki process and uses the timer library.

Callback timers (ctimer library) provide a timer mechanism that calls a specified
C function when a ctimer expires~\cite{contiki-docs}.
%Like event timers, they are used to wait for some time while the rest
%of the system can work or enter low power mode.
Thus, they are especially useful in any code that does not have an
explicit Contiki process such as protocol implementations~\cite{contiki-wiki-timers}.

The Real-time timers (rtimer library) handle the scheduling and execution of
real-time tasks with predictable execution times~\cite{contiki-docs}.
The rtimer library provides real-time task support through callback functions -
the rtimer immediately preempts any running Contiki process in order to let the real-time tasks
execute at the scheduled time~\cite{contiki-wiki-timers}.
This behaviour is illustrated in figure~\ref{fig:contiki-execution-context}.
The rtimer library uses a separate hardware clock
to allow higher clock resolution~\cite{contiki-wiki-timers}.
The small part of the rtimer library is architecture-agnostic,
but the particular implementation is platform-specific.
