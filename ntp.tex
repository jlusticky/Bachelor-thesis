%=========================================================================
% (c) 2011, 2012 Josef Lusticky <xlusti00@stud.fit.vutbr.cz>

\chapter{Network Time Protocol}
Network Time Protocol provides mechanism for synchronising systems' clocks over the variable-latency data network.
NTP was introduced and is still developed by David Mills at University of Delaware in Newark, United States.
NTP is argueably the longest running, continuously operating,
ubiquitously available protocol in the Internet~\cite{ntp-overview}.
Despite being one of the oldest surviving protocol on the Internet it is not old-fashioned at all.
NTP version 4 described in RFC~5905~\cite{rfc5905} is an update to older NTPv3 to accomodate NTP to IPv6.
Version 4 also includes fundamental improvements in
the mitigation and discipline algorithms that extend
the potential accuracy to the tens of microseconds with modern
workstations and fast LANs~\cite{rfc5905}.
NTPv4 corrects some
errors in NTPv3 design and includes optional extension mechanism
that can be used for adding more capabilites to NTP, e.g. the
Autokey security protocol described in RFC~5906~\cite{rfc5906}
for authenticating servers to clients.

Simple Network Time Protocol is simplified NTP implementation lacking complex
synchronisation algorithms used by NTP. SNTP is also described in RFC 5906~\cite{rfc5906}.

SNTP is a simplified sub-set of the algorithms used by the NTP protocol.


\section{Time and timestamps}

A primary server is synchronized to a reference clock directly traceable
to UTC~\cite{rfc5905}.

\! 
\! rfc5905
\! older "Time" protocol


The Coordinated Universal Time (UTC) timescale represents mean solar
   time as disseminated by national standards laboratories.  The system
   time is represented by the system clock maintained by the hardware
   and operating system.  The goal of the NTP algorithms is to minimize
   both the time difference and frequency difference between UTC and the
   system clock.  When these differences have been reduced below nominal
   tolerances, the system clock is said to be synchronized to UTC.



   The date of an event is the UTC time at which the event takes place.
   Dates are ephemeral values designated with uppercase T.  Running time
   is another timescale that is coincident to the synchronization
   function of the NTP program~\cite{rfc5905}.



\section{Packet structure}
The 64-bit timestamps used by NTP consist of a 32-bit seconds part and a 32-bit fractional second part.

\section{Network}\label{sec:ntp-network}
Network specification of NTP 
The protocol uses the User Datagram Protocol (UDP) on port number 123~\cite{ianna-ports}.
The NTP packet is a UDP datagram, carried on port 123.

\section{System}
