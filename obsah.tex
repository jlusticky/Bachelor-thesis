%=========================================================================
% (c) 2011, 2012 Josef Lusticky

\chapter{Introduction}
Today we live in a world where embedded systems are part of almost every electronic device.
Modern televisions contain embedded systems to allow you to browse the Web,
modern cars use embedded systems to control an engine or to give you summary
about your journey using GPS, even fridges showing the list of things you should buy at a market are becoming popular.
Embedded systems are becoming spread more than ever and so does
their need for a network connection.

Contiki is an operating system targeted at embedded systems
developed by Adam Dunkels at Swedish Institute of Computer Science in Kista, Sweden.
Contiki brings new concepts to the embedded world such as Protothreads and features
Internet Protocol version 6 and 4 support.
Since Contiki aims for maximum portability, it is written in the C programming language.
Contiki therefore provides an ideal solution for connecting
embedded systems to an existing network on many different hardware platforms.

Time synchronisation is also important nowadays.
Almost every modern system has a need for an exact time;
your video-recorder or home cinema automatically starts recording a film at a scheduled time,
your washing-machine should have finished the
desired program when you return home or your radio automatically adjusts its clock when the time changes
due to daylight saving.

Network Time Protocol (NTP) is the ubiquitous protocol for time synchronisation between computers in modern Internet.
Though being one of the oldest protocols, NTP is still developed and updated to conform to the latest
network standards. Actual version at the time of writing is NTP version~4, which updates its previous version to
accommodate the Internet Protocol version~6.

This thesis describes Contiki operating system, its concepts and philosophy,
Network Time Protocol version~4 and design and implementation of NTP client for Contiki operating system.


%=========================================================================
% (c) 2011, 2012 Josef Lusticky

\chapter{Contiki OS}
Only 2\% of all microprocessors that are sold today are used in PCs and the remaining 98\%
of all microprocessors are used in embedded systems~\cite{thesis-programming}.
Embedded systems have much smaller amounts of memory than PC computers.
Moore's law predicts that these devices
can be made significantly smaller and less expensive in the future.
While this means that embedded system networks can
be deployed to greater extents, it does not necessarily imply
that the resources will be less constrained~\cite{paper-contiki}.
The memory constraints make programming for embedded systems a challenge.

Operating system Contiki is targeted at embedded systems supporting MSP430, AVR, ARM, x86
architectures and many others~\cite{contiki-docs}.
Contiki aims for maximum portability and therefore is written in C.
It is a feature-rich operating system and
only some of its features are described in this thesis.

Contiki is developed by a group of developers from industry and academia
lead by Adam Dunkels from the Swedish Institute of Computer Science.
The Contiki team currently consists of sixteen developers from SICS,
SAP AG, Cisco, Atmel, NewAE and TU Munich~\cite{contiki-docs}.
Contiki is also deployed at RheinMain University in Wiesbaden.
The 3-clause BSD license places minimal restrictions on redistribution of Contiki.
Version 1.0 of Contiki OS was released in 2002, version 2.0 in 2007 and the latest version
at the time of writing was 2.5 released in 2011.
The actual development happens in online repository accessible on Contiki homepage at \url{http://www.contiki-os.org/}
using Git version control system.

%=========================================================================
% (c) 2011, 2012 Josef Lusticky

\section{Features}
Contiki OS features lightweight stackless threads called Protothreads.
Protothreads are a new concept brought to the embedded world by Contiki,
they are extremely lightweight and compatible with standard C~\cite{paper-protothreads}.
Each Protothread does not require a separate stack which fits them perfectly
for usage in memory constrained embedded systems.
Protothreads are more detailed discussed in section~\ref{sec:contiki-protothreads}.

Contiki also features TCP/IP communication stack called uIP~(micro~IP)
that conforms to Request For Comments memorandums published by the Internet Engineering Task Force.
The uIP provides communication abilities using both IPv4 and IPv6~\cite{contiki-docs}.
Contiki with its uIP stack is IPv6 Ready Phase 1 certified
and therefore has right to use the IPv6~Ready silver logo~\cite{ipv6ready-db}.
Before Contiki's uIP, the embedded world considered IP to be too heavyweight.
That means all earlier IP implementations for general purpose computers
were much bigger than the memory constrained embedded systems could use~\cite{interconnecting}.
The communication stack uIP is closely described in section~\ref{sec:contiki-uip}.

Next to the uIP, Contiki is equipped with other communication stack called Rime.
Rime is a layered communication stack for sensor networks,
with much thinner layers than traditional architectures~\cite{paper-rime}.
Rime is designed to simplify the implementation of communication
protocols on low-power radios.
The communication primitives in the Rime stack were chosen
based on what typical sensor network protocols use -
single-hop unicast, single-hop broadcast or multi-hop~\cite{contiki-docs,paper-rime}.

Beside Protothreads, uIP and Rime,
Contiki contains a very simple, relatively small and easy to use filesystem
called Coffee Filesystem (CFS),
a graphical user interface called Contiki Toolkit (CTK),
Executable Linkable Format (ELF) loader for loading object files into a running Contiki system
and much more.

Operating system Contiki, uIP and Protothreads are used by hundreds of companies in embedded devices in
such diverse systems as car engines, oil boring equipment, satellites, and container security systems~\cite{thesis-programming}.
The software is also used both in academic research
projects and in university project courses on embedded systems throughout the
world.

\begin{figure}
  \centering
  \includegraphics[width=9cm,keepaspectratio]{fig/contiki-vnc.png}
  \caption{Screenshot of running Contiki OS with CTK (source:~\cite{contiki-docs})}
\end{figure}


%=========================================================================
% (c) 2011, 2012 Josef Lusticky <xlusti00@stud.fit.vutbr.cz>

\section{Protothreads}\label{sec:contiki-protothreads}
Protothreads provide a way to run C functions quasi-paralelly, that is, a C functions work in a way similiar to thread.
In Contiki Protothreads allow process to wait for incoming events. While waiting for an event to occur another function
can be run. The core of this solution is C switch statement used in conjuction with variable (called local continuation)
containing the position where the function was interrupted. Next time function continues from this point.

The advantage of Protothreads over ordinary threads is that a Protothread does not require a separate stack.
In memory constrained systems, the overhead of allocating multiple stacks can consume large amounts of
the available memory. In contrast, each Protothread only requires few bytes for storing the state of execution.

A Protothread is driven by repeated calls to the function in which the Protothread is running.
Each time the
function is called, the Protothread will run until it blocks or exits.
Protothreads are implemented using local continuations. A local continuation represents the current state
of execution at a particular place in the program, but does not provide any call history or local variables.

The Protothreads API consists of four basic operations: initialization (PT\_INIT()), execution (PT\_BEGIN()),
conditional blocking (PT\_WAIT\_UNTIL()) and exit (PT\_END()). On top of these, two convenience functions
are built: reversed condition blocking (PT\_WAIT\_WHILE()) and Protothread blocking (PT\_WAIT\_THREAD())~\cite{paper-protothreads}.

To understand how are Protothreads implemented and how do the actually work please refer
to appendix~\ref{app:protothreads} in which an example of usage is shown.

Since Protothreads are implemented using standard C, library providing Protothreads can be used everywhere C toolchain is available.
But there are some cons to consider. Because protothreads are stackless, a Protothread can only run within a single C function.
There is also no way of storing automatic local variables. And since Protothreads are implemented using C {\it switch} statement, and these can
not be nested, the code that uses Protothreads cannot use {\it switch} statements itself.
Workaround for storing local variables is to prepend them with the {\it static} keyword, which make them being put into data segment
by compiler and thus remembering the value between the function calls.


%=========================================================================
% (c) 2011, 2012 Josef Lusticky

\section{uIP}\label{sec:contiki-uip}
The TCP/IP protocol suite is often used for communication over the Internet as well as local networks.
uIP (micro IP) is a complete TCP/IP communication stack developed by Adam Dunkels
for memory constrained systems such as embedded systems.

Before uIP, the TCP/IP architecture was considered to be heavyweight
due to its perceived need for processing power and memory.
IP protocol was seen as too large to fit into the constrained environment -
existing implementations of the IP protocol family for general purpose computers would need hundreds
of kilobytes, whereas a typical constrained system has only a few tens of kilobytes of memory~\cite{interconnecting}.
For this reason, several non-IP stacks were developed.

In early 2000s, however, this view was challenged by lightweight implementations of the IP
protocol family for smart objects such as the uIP stack~\cite{interconnecting}.
uIP showed that the IP architecture would fit nicely into the typical constrained systems,
without removing any of the essential mechanisms from IP.
Note that these resources, which are considered constrained today, are fairly close to the
resources of general purpose computers that were available when IP was designed~\cite{interconnecting}.
Since its initial release, the uIP stack has become widely used in networked
embedded systems~\cite{interconnecting, thesis-programming}.

uIP provides two different application programming interfaces to programmers:
a BSD sockets-like API called Protosockets and raw event-driven API.
Protosockets are based on Protothreads putting the same limitation on them - such as
no way of storing automatic local variables and an impossibility of using the C {\it switch} statement.
Protosockets only work with TCP connections~\cite{contiki-docs}.
Since NTP uses UDP, Protosockets will not be further
discussed in this thesis. For more information about Protosockets
please refer to Contiki documentation~\cite{contiki-docs}.

uIP contains only the absolute minimum of required features to fulfill the protocol standard.
It can only handle a single network interface and contains the IP, ICMP, UDP and TCP protocols~\cite{contiki-docs}.
In order to reduce memory requirements and code size,
the uIP implementation uses an event-based API, which is fundamentally different
from the most common TCP/IP API, the BSD sockets API, present on Unix-like systems
and defined by POSIX standard~\cite{thesis-programming,posix}.
An application is invoked in response to certain events and
it is up to the application that is receiving events from uIP to handle all
work with data to be transmitted. E.g. if the data is lost in the network,
the application will be invoked and then has to resend the data.
This approach is based on the fact that it should be an easy work for application
to rebuild the same data.
This way uIP stack does not need to use explicit dynamic memory allocation.
Instead it uses a single global buffer for holding packets and has a fixed
table for holding connection state.
The global packet buffer is large enough to contain one packet of maximum size~\cite{contiki-docs}.

When a packet arrives from network, the device driver places it in the
global buffer and calls the TCP/IP stack.
If the packet contains data, the TCP/IP stack will notify the corresponding application.
Because the data in the buffer will be overwritten by the next incoming packet,
the application will either have to act immediately on the data or copy the data into
its own buffer for later processing.
The packet buffer will not be overwritten by new packets before the application has processed the data~\cite{contiki-docs}.
Packets that arrive when the application is processing the data must be queued,
either by the network device or by the device driver.
That means uIP relies on the hardware when it comes to buffering.
Most single-chip Ethernet controllers have on-chip buffers
that are large enough to contain at least 4 maximum sized Ethernet frames~\cite{contiki-docs}.
This way uIP does not have to have its own buffer structures and thus
needs only minimal memory amount.
Possible packet loss is a trade-off for minimalism and ability to communicate using TCP/IP.
It is not such a big deal for communication using TCP on transport layer
because of acknowledgement scheme used in TCP to prevent data loss.
However data carried on UDP can be irrecoverably lost.

As was expected, measurements show that uIP implementation provides very low
throughput, particularly when communicating with a PC host~\cite{thesis-towards}.
However small systems that uIP is targeting usually do not produce enough data
to make the performance degradation a serious problem~\cite{thesis-towards}.

Despite being so small uIP is not only RFC compliant but also IPv6 Ready Phase 1 certified.
uIP is written in C programming language and it is fully integrated to Contiki operating system.
In uIP, there are even some more tricks to shrink the stack
but complete uIP description is outside the scope of this thesis.
Please refer to Contiki documentation for more details~\cite{contiki-docs}.


%=========================================================================
% (c) 2011, 2012 Josef Lusticky

\section{Kernel and processes}\label{sec:contiki-kernel}
The kernel in Contiki is event-driven providing cooperative multitasking
environment, but the system supports preemptive
multithreading that can be applied on a per-process basis~\cite{video}.
The preemption is not implemented in the kernel, but
preemptive multithreading is implemented as a library that is linked only with programs that
explicitly require multithreading~\cite{paper-contiki}.
The kernel itself contains no platform specific code, it implements only CPU multiplexing and
lets device drivers and applications communicate directly with hardware~\cite{video}.

From high level of abstraction,
applications in Contiki OS are implemented and run as processes.
Protothreads, the lightweight threads described in section~\ref{sec:contiki-protothreads},
are used in Contiki to implement processes.
Both the Contiki kernel and Contiki applications use
Protothreads extensively to achieve cooperative multitasking~\cite{contiki-wiki-faq}.
Every Contiki process consists of a process control block and a process thread~\cite{contiki-wiki-processes}.
The process control block contains run-time information about the process and
the process thread contains the code of the process.
Among other things, the process control block contains
textual name of the process, pointer to the process thread and state of the process.
The process thread is implemented as a single Protothread,
that is invoked from the process scheduler in the Contiki kernel~\cite{contiki-wiki-processes}.

From low level of abstraction,
every application is implemented as a simple C function
and the process control block remembers the actual state of execution of this function
in the same way as the local continuation works by Protothreads.
Processes are therefore running quasi-parallely in Contiki.

\bigskip
\begin{lstlisting}[caption=Process control block in Contiki OS]
struct process {
	struct process *next;
	const char *name;
	int (* thread)(struct pt *, process_event_t, process_data_t);
	struct pt pt;
	unsigned char state, needspoll;
	};
\end{lstlisting}

Process control block is not declared or defined directly,
but through the {\it{PROCESS()}} macro.
This macro takes two parameters: the variable name of the process control block
and a textual name of the process,
which is used in debugging and when printing out lists of active processes to users~\cite{contiki-wiki-processes}.

All code execution is initiated by the Contiki kernel
that acts like a simple dispatcher calling these functions~\cite{contiki-docs}.
Just like Protothreads, processes are also implemented using macros,
making them fully standard C compatible.

In Contiki, code run in either of two execution contexts:
cooperative, in which code never preempts other code, and preemptive,
which preempts the execution of cooperative code and returns control
when preemptive code is finished.
Processes always run in cooperative mode,
whereas interrupt service routines and real-time timers run in preemptive mode~\cite{contiki-wiki-processes}.
Code running in both execution contexts illustrates figure~\ref{fig:contiki-execution-context}.

\begin{figure}
  \centering
  \includegraphics[width=13cm,keepaspectratio]{fig/Execution-contexts.png}
  \caption{Contiki execution contexts by A. Dunkels}
  \label{fig:contiki-execution-context}
\end{figure}

Interprocess communication is done by posting events in Contiki OS -
processes communicate with each other by posting events to each other~\cite{paper-contiki}.
There are two types of events: synchronous and asynchronous.
Synchronous events are directly delivered to the receiving process when posted and
can only be posted to a specific processes~\cite{contiki-wiki-processes}.
Because synchronous events are delivered immediately,
posting synchronous event is equivalent to a function call:
the process to which an event is delivered is directly invoked,
and the process that posted the event is blocked
until the receiving process has finished processing the event~\cite{contiki-wiki-processes}.

Asynchronous events are delivered to the receiving process
some time after they have been posted~\cite{contiki-wiki-processes}.
Before delivery, the asynchronous events are held on event queue inside Contiki kernel.
The kernel loops through this event queue and delivers
the event to the process by invoking the process. 
The receiver of an asynchronous event can be either a specific process
or all running processes~\cite{contiki-wiki-processes}.

%! paper-contiki - dunkels04contiki
%Being able to power down the device
%when the network is inactive is often required way to reduce energy consumption.
%Power conservation mechanisms
%depend on both the applications and the network protocols.
%The Contiki kernel contains no explicit power
%save abstractions, but lets the the application specific parts
%of the system implement such mechanisms.
%To help the application decide when to power down the system, the event
%scheduler exposes the size of the event queue.
%This information can be used to power down the processor when there
%are no events scheduled.
%! paper-contiki

%As stated before, Contiki is well documented and you can find out more about
%the kernel as well as the system in the documentation~\cite{contiki-docs}.


%=========================================================================
% (c) 2011, 2012 Josef Lusticky

\section{Timers}\label{sec:contiki-timers}
The Contiki kernel does not provide support for timed events,
instead an application that wants to use timers needs to explicitly use a timer library.
The timer library provides functions for setting, resetting and restarting timers,
and for checking if a timer has expired.
An application must manually check if its timers have expired - this is not done automatically~\cite{contiki-docs}.

Contiki has one clock library and a set of timer libraries: timer, stimer, ctimer, etimer, and rtimer~\cite{contiki-wiki-timers}.
The clock library provides functionality to handle the system time and also to block the CPU for short time periods.
It is the interface between Contiki and the platform specific clock functionality~\cite{contiki-docs}.
The timer libraries are implemented with the functionality of the clock library as a base~\cite{contiki-wiki-timers}.

The timer and stimer libraries provide the simplest form of timers and are used to check if a time period has passed.
The difference between these two is the resolution of time -
timers use system clock ticks, whose value is incremented when an interrupt from the hardware clock occurs,
while stimers use seconds to offer much longer time periods~\cite{contiki-wiki-timers}.
The value representing seconds is also incremented in the interrupt service routine (ISR),
but only when enough clock ticks since last increment occurred.
The number of clock ticks within one second is represented by the
{\it{CLOCK\_SECOND}} macro provided by the clock library.
That means there are {\it{CLOCK\_SECOND}} interrupts from the hardware clock per second.
The usage of the timer library and {\it{CLOCK\_SECOND}} macro is shown in appendix~\ref{app:protothreads}.

The simplest timer and stimer libraries are not able to post an event when a timer expires.
Event timers should be used for this purpose.
Event timers (etimer library) provide a way to generate timed events.
An event timer will post an event to the process that set the timer when the
event timer expires~\cite{contiki-docs}.
The etimer library is implemented as a Contiki process and uses the timer library.

Callback timers (ctimer library) provide a timer mechanism that calls a specified
C function when a ctimer expires~\cite{contiki-docs}.
%Like event timers, they are used to wait for some time while the rest
%of the system can work or enter low power mode.
Thus, they are especially useful in any code that does not have an
explicit Contiki process such as protocol implementations~\cite{contiki-wiki-timers}.

The Real-time timers (rtimer library) handle the scheduling and execution of
real-time tasks with predictable execution times~\cite{contiki-docs}.
The rtimer library provides real-time task support through callback functions -
the rtimer immediately preempts any running Contiki process in order to let the real-time tasks
execute at the scheduled time~\cite{contiki-wiki-timers}.
This behaviour is illustrated in figure~\ref{fig:contiki-execution-context}.
The rtimer library uses a separate hardware clock
to allow higher clock resolution~\cite{contiki-wiki-timers}.
The small part of the rtimer library is architecture-agnostic,
but the particular implementation is platform-specific.



%=========================================================================
% (c) 2011, 2012 Josef Lusticky <xlusti00@stud.fit.vutbr.cz>

\chapter{Network Time Protocol}
Network Time Protocol provides mechanism for synchronising systems' clocks over the variable-latency data network.
NTP was introduced and is still developed by David Mills at University of Delaware in Newark, United States~\cite{ntp-history}.
NTP is argueably the longest running, continuously operating,
ubiquitously available protocol in the Internet~\cite{ntp-overview}.
Despite being one of the oldest surviving protocol on the Internet, it is not old-fashioned at all.
NTP version 4 described in RFC~5905~\cite{rfc5905} is an update to older NTPv3 to accomodate NTP to IPv6.
Version 4 also includes improvements in
the mitigation and discipline algorithms that extend
the potential accuracy to the tens of microseconds with modern
workstations and fast LANs~\cite{rfc5905}.
NTPv4 corrects some
errors in NTPv3 design and includes optional extension mechanism
that can be used for adding more capabilites to NTP, e.g. the
Autokey security protocol described in RFC~5906
for authenticating servers to clients.

Simple Network Time Protocol is simplified NTP implementation lacking complex
synchronisation algorithms used by NTP~\cite{rfc5905}.
SNTP is also described in RFC 5905.
The packet of SNTP has the same structure and content as packet of NTP~\cite{rfc5905}.
From observing the network communication one can not tell whether the client
is full blown NTP implementation or just SNTP.
SNTP is a simplified sub-set of the algorithms used by the NTP protocol
making the client implementation not only easier, but also suitable for
resource constraint systems such as embedded systems.
Since NTP and SNTP servers and clients are
completely interoperable and can be intermixed in NTP subnets~\cite{rfc5905},
this thesis refers to SNTP client for Contiki OS as NTP client.


%=========================================================================
% (c) 2011, 2012 Josef Lusticky <xlusti00@stud.fit.vutbr.cz>

\section{Topology and hierarchy}\label{sec:ntp-topology}
NTP uses two different communication modes:
one to one, referred as unicast, and one to many, referred broadcast~\cite{rfc5905}.
In unicast communication mode, NTP client sends request and NTP server sends response.
In broadcast communication mode, the client sends no request
and waits for a broadcast message from one or more servers~\cite{rfc5905}.

NTP servers are rated with stratum (plural form strata) number which represents their position
in an NTP hierarchy and their possible accuracy~\cite{rfc5905}.
Primary (stratum 1) servers synchronise to the reference clock directly traceable to UTC via
radio, satellite or modem.
The stratum 2 servers synchronise to stratum 1
servers via hierarchical subnet.
The stratum 3 servers synchronise to stratum 2 servers, and so on.
The maximum stratum is 15, number 16 means unsynchronised server
and higher numbers (up to 255) are reserved~\cite{rfc5905}.
Synchronisation between servers in the same stratum level is also possible.
Figure~\ref{fig:ntp-hierarchy} shows a brief hierarchy of NTP.
\begin{figure}
  \centering
  %\input{./xfig/test.pstex_t}
  \includegraphics[width=9cm,keepaspectratio]{fig/Network_Time_Protocol_servers_and_clients.pdf}
  \caption{Topology and hierarchy of NTP by B. Esham}
  \label{fig:ntp-hierarchy}
  \bigskip
\end{figure}


%=========================================================================
% (c) 2011, 2012 Josef Lusticky

\section{Time and timescales}\label{sec:ntp-time}
For expressing the time NTP always uses the Coordinated Universal Time (UTC)~\cite{rfc5905}.
UTC is maintained by the International Bureau of Weights and Measures in Paris, France.
It is the time scale that forms the basis for coordinated dissemination
of standard frequencies and time signals~\cite{bipm-utc}.
The time specified by UTC is the same for all timezones.
Its calculation is the same as with Greenwich Mean Time (GMT),
however the daylight savings are not accounted.

The UTC scale is adjusted by the insertion of leap seconds to ensure approximate
agreement with the time derived from the rotation of the Earth~\cite{bipm-utc}.
The atomic clocks, on which UTC is based, are so precise that
they do not match the rotation of the Earth,
which periodically speeds up and slows down due to the action
of tides and changes within the Earth's core~\cite{cnn-earth}.
The goal of a leap second is to catch up UTC with these changes.
The leap second is inserted or deleted on the advice of
International Earth Rotation and Reference Systems Service~\cite{bipm-utc}.
NTP is well designed for leap second occurrence -
there is Leap Indicator field
in the structure of NTP packet and there are also fields intended for
information about leap second in structures that NTP algorithm uses~\cite{rfc5905}.
The formal definition of UTC does not permit double leap seconds~\cite{posix}.

In computer the system time is represented by system clock maintained by
hardware and operating system.
The goal of the NTP algorithms is to minimize
both the time difference and frequency difference between UTC and the system clock.
When these differences have been reduced below nominal
tolerances, the system clock is said to be synchronised to UTC~\cite{rfc5905}.
It has never been a goal of NTP to take care of local time,
it is up to operating system to provide user the correct local time~\cite{ntp-overview}.

The NTP and POSIX timescales are based on the UTC timescale,
but not always coincident with it~\cite{ntp-leap}.
Both timescales reckon in seconds since the prime epoch,
but the origin of the NTP timescale, the NTP prime epoch, is 00:00:00 UTC on 1 January 1900,
while the prime epoch of the POSIX timescale is 00:00:00 UTC on 1st January 1970~\cite{ntp-leap}.
So upon the first tick of POSIX clock on 1st January 1970 the NTP clock read 2~208~988~800,
representing the number of seconds since the NTP prime epoch.


%=========================================================================
% (c) 2011, 2012 Josef Lusticky <xlusti00@stud.fit.vutbr.cz>

\section{Network and timestamps}\label{sec:ntp-network}
Network specification of NTP defines that
the protocol uses the User Datagram Protocol (UDP) on port number 123~\cite{ianna-ports,rfc5905}.
Reliable message delivery such as TCP can actually make the delivered of
NTP packet less reliable since retries
would increase the delay value and other errors~\cite{rfc5905}.
This is mostly due to overhead of communication with TCP on transport layer.

NTP manipulates with the time through timestamps - a record of time.
NTP timestamp has two fields: the seconds field expressing the number of seconds
and the fraction field expressing fraction of a second~\cite{rfc5905}.
All NTP time values are represented in twos-complement format, with
bits numbered in big-endian fashion from zero starting at the left, or high-order, position~\cite{rfc5905}. 
There are two formats of timestamp in NTP packet structure:
long 64-bit and short 32-bit as shown on figure~\ref{fig:ntp-timestamps}.
The 64-bit long timestamp used by NTP consists of a 32-bit unsigned seconds
field spanning $2^{32}$ seconds (aprox. 136 years from 1900 to 2036) and a 32-bit fraction field resolving
$2^{-32}$ seconds (aprox. 232 picoseconds)~\cite{rfc5905}.
The short 32-bit timestamp includes a 16-bit unsigned seconds field
and 16-bit fraction field.

There is one more NTP time format - 128-bit NTP Date format.
It includes a 64-bit signed seconds field and 64-bit fraction field.
For convenience in mapping between formats,
the seconds field is divided into a 32-bit Era Number field
and a 32-bit Era Offset field.
This 128-bit NTP Date format is however not transmitted over the network
and is only used where sufficient storage and word
size is available~\cite{rfc5905}.
So there is practically no need of knowing about this format for embedded systems
at least until year 2036, when the Era Number will be incremented from zero to one.
But strictly speaking an NTP timestamp is a truncated NTP Date format~\cite{rfc5905}.
Refer to apendix~\ref{app:dates} for a short list of some dates
requiring usage of NTP Date format.

\begin{figure}
	\centering
	\includegraphics[width=13cm,keepaspectratio]{fig/ntp-timestamps.pdf}
	\caption{Time formats used in NTP packet}
	\label{fig:ntp-timestamps}
	\bigskip
\end{figure}

% TODO - packet description

\begin{figure}
	\centering
	\includegraphics[width=9cm,keepaspectratio]{fig/ntp-packet.pdf}
	\caption{Basic NTP packet structure by D. Mills}
	\label{fig:ntp-packet}
	\bigskip
\end{figure}

Because Root dispersion and Root Delay fields do not need so big scope and precision,
the short 32-bit format is used for them.
Root dispersion express accumulated total dispersion from primary server
and Root Delay express accumulated roundtrip delay via primary server~\cite{ntp-arch}.


%TODO

Because of network latency the timestamp recieved will never be exactly corresponding to
the current time.
One of the main goals of NTP is to deal with the network latency~\cite{ntp-overview}.


%=========================================================================
% (c) 2011, 2012 Josef Lusticky <xlusti00@stud.fit.vutbr.cz>

\section{Algorithms}
As the figure~\ref{fig:ntp-packet} shows, there are four 64-bit long timestamps
in NTP packet: Reference, Origin, Receive and Transmit timestamp.
%The Reference timestamp is when server was synchronised
%! TODO ->
Using these four timestamps, NTP client can compute
the local clock offset which is given by $\theta = \frac{1}{2}[(t_2 - t_1) + (t_3 - t_4)]$,
where $t_1$ is the time of the request packet transmission,
$t_2$ is the time of the request packet reception,
$t_3$ is the time of the response packet transmission and
$t_4$ is the time of the response packet reception~\cite{ntp-algor,rfc5905}.
%! t_4 is not packet XX
%! figure

%! communication with one server - RFC958
The destination peer calculates the roundtrip delay and clock
      offset relative to the source peer as follows.  Let t1, t2 and t3
      represent the contents of the Originate Timestamp, Receive
      Timestamp and Transmit Timestamp fields and t4 the local time the
      NTP message is received.  Then the roundtrip delay d and clock
      offset c is:

         d = (t4 - t1) - (t3 - t2)  and  c = (t2 - t1 + t3 - t4)/2 .

      The implicit assumption in the above is that the one-way delay is
      statistically half the roundtrip delay and that the intrinsic
      drift rates of both the client and server clocks are small and
      close to the same value.
%one server

When computing result from more servers the intersection algorithms is used
for selecting the possible most exact timestamp received from various servers~\cite{rfc5905}.
% Derived from Murzollo algorithm but the base remains... ~\cite{ntp-history}.
The resulting exact timestamp does not have to be the same
as one of those servers provided.
First of all a set of bad and good servers must be made.
Bad servers are called Falsetickers and good are called Truechimers~\cite{rfc5905}.
The division to these sets is based on their response.
As one can assume for sensible result there must be more Truechimers than Falsetickers~\cite{rfc5905}.

Intersection algorithm computes with estimates converted to intervals.
Figure~\ref{fig:ntp-intersection} shows the computation for the following example:
If we have the estimates $10 \pm 2$, $12 \pm 1$ and $11 \pm 1$
then these intervals are $<8; 12>$, $<11; 13>$ and $<10; 12>$ which
intersect to form $<11; 12>$ or $11.5 \pm 0.5$ as consistent with all three values.
The arithmetic mean is used as a value of result.
When querying servers again, the algorithm repeats but the new result computation
also depends on the previous result~\cite{rfc5905}.
This eliminates possible jitter which can be caused by repeatedly querying the servers
and getting slightly different answers from them.

\begin{figure}
	\centering
	\includegraphics[width=13cm,keepaspectratio]{fig/Marzullo_example-1.jpg}
	\caption{Intersection algorithm by D. Exb}
	\label{fig:ntp-intersection}
	\bigskip
\end{figure}

%Since the clients complying with a subset of NTP, called
%the Simple Network Time Protocol (SNTPv4) [RFC4330], do not need to
%implement the mitigation algorithms ... ~\cite{rfc5905}.


%! review
\section{Hardware concerns for implementing real-time support}
A typical desktop computer today includes CPU based on Intel x86 architecture.
Real-Time Clock (RTC) in CMOS memory that is battery powered

Unfortunately Intel x86 architecture is heavily influenced by backwards compatibility,
e.g. the time value can also be stored in Binary Code Digit (BCD) encoding in RTC.

In year 19xx / Starting with Intel 386
Intel introduced
Programmable Interrupt Controller (PIT) Intel 8253 and 8254 - 3 counters (counter 0 interrupt to OS)


Used by historic versions of Linux
=> read initial time from RTC, setup PIT and interrupts (IRQ 0, INT 8), increment jiffies on every interrupt, provide app resolution of jiffies

init/main.c - time\_init() - read from RTC and save to startup\_time
kernel/sched.c - sched\_init() = PTI setup for interrupts - LATCH (1193180/HZ)
kernel/system\_call.s - timer\_interrupt() in assembly - increments jiffies

The current real time is provided by CURRENT\_TIME (startup\_time+jiffies/HZ) => since jiffies is integer and HZ is 100 => resolution of 10ms.
kernel/sys.c - sys\_time() - CURRENT\_TIME returned


\section{NTP on POSIX-compliant systems}
Operating systems
Linux
OpenBSD
DragonflyBSD

NTP implementations
NTP from ntp.org project - reference implementation
Chrony
OpenNTPD - OpenBSD
dntpd - DragonflyBSD



%=========================================================================
% (c) 2011, 2012 Josef Lusticky

\chapter{Analysis}
For implementation of a reasonably useful NTP client,
an operating system must be able to set, get and eventually adjust the system time.
Though not mandatory, adjusting the time is an important function,
if the time shall be always a monotonically increasing function.
Apart from that, ability to communicate over UDP is also required.

For developing and testing Contiki NTP Client,
the AVR Raven platform with 8-bit ATmega1284P CPU~\cite{avr-datasheet} will be used.
This platform features IEEE~802.15.4 (Low-Rate Wireless Personal Area Networks) link layer support.
Together with an adaptation layer called 6LoWPAN (IPv6 over Low power Wireless Personal Area Networks),
AVR Raven is able to communicate over IPv6.

%=========================================================================
% (c) 2011, 2012 Josef Lusticky

\section{Time interface}\label{sec:analysis-time}
The low-level clock interface described in previous section
is used by Contiki to provide the system time through the time interface.
Since the value of the {\it{seconds}} variable is zero after the system booted,
it actually represents the system uptime.
The value of the {\it{seconds}} variable can be obtained by the application using the {\it{clock\_seconds}} call.
However, there is no call for setting this variable in Contiki 2.5.
In the current Git version at the time of writing, a new call {\it{clock\_set\_seconds}}
can be used for this purpose.
Because this call simply rewrites the {\it{seconds}} variable, it breaks the stimer library,
and should be therefore avoided by the NTP client.
Similarly, setting the {\it{scount}} variable would cause
unbalanced increments of the {\it{seconds}} variable.

The precision of one second is also not adequate for the NTP client.
Further precision can be acquired by reading the {\it{scount}} variable,
as it provides a resolution of $\frac{1}{CLOCK\_SECOND}$~seconds.
Moreover, the hardware counter can be also queried, as it includes the time passed since
the last update of the {\it{scount}} variable.
If stimers should not be broken by setting the {\it{seconds}} or {\it{scount}} variable,
and Contiki should be able to provide the current time in a higher precision,
a new call interface must be designed.
This call interface shall use a time specification structure similar
to the existing POSIX standard~\cite{posix}.
Such a structure for representing the time values is also not present in Contiki.

Similarly, there is no call for adjusting the time in Contiki.
Due to memory constraints, software structures controlling the time adjustments are too heavyweight
for use in an embedded operating system running on 8-bit CPUs.
Due to low CPU frequencies, the code of an interrupt service routine can not be complex
and sophisticated clock discipline algorithms should be avoided.
A call for adjusting the time should therefore use abilities
provided by the hardware clock as much as possible.

Updating the value in the {\it{OCR2A}} compare register
can be used for adjusting the time, because decrementing the compare register
value causes a faster increment of the {\it{scount}} variable, which in turn causes
a faster increment of the {\it{seconds}} variable and vice versa.
Such changes would influence the system time and the dependent Contiki timers.
However, applications requiring uninfluenced timers
could use the Contiki rtimers, described in section~\ref{sec:contiki-timers},
because they use a separate hardware clock unaffected by these changes
(Timer/Counter~3 on AVR Raven platform).


%=========================================================================
% (c) 2011, 2012 Josef Lusticky

\section{Clock subsystem}\label{sec:analysis-clock}
Contiki provides basic clock interface particularly for use of timers
with a simple goal - measuring time.
This interface is common for all supported platforms,
but the particular implementation is platform specific.
The common interface definition is located in {\it{core/sys/clock.h}} file
and the specific implementations can be found in {\it{clock.c}} file
in {\it{cpu/}} directory of Contiki source code.

The clock interface provides {\it{clock\_init}} call for initialising the hardware clock,
that is automatically called during boot sequence of Contiki.
The {\it{clock\_init}} call sets up
appropriate counter registers and interrupt service routines as described in section~\ref{sec:analysis-clock}.
This call is implemented as a macro for AVR CPUs, which evaluates to a specific setup code for each
different type of AVR CPU during compilation, and is defined in {\it{cpu/avr/dev/clock-avr.h}} file.
The setup code is not common to all AVR CPUs because of differences among them - e.g. there are usually
only three Timer/Counter modules, but AVR ATmega1284P has four Timer/Counter modules~\cite{avr-datasheet}.

On AVR Raven, 8 bit Timer/Counter~2 clocked from asynchronous 32~768~Hz crystal oscillator
is used by default by Contiki clock interface.
This oscillator is independent of any other clock,
can be only used with Timer/Counter~2 and it
enables use of Timer/Counter~2 as a Real Time Counter~\cite{avr-datasheet}.
The Timer/Counter~2 prescale value 8 is used in Contiki on AVR Raven platform,
so that oscillator frequency 32~768~Hz is effectively divided by 8.
Counter register is hence incremented with frequency
$f_{T2} = {\frac{f_{asy}}{prescaler}} = {\frac{32768}{8}} = 4096$ Hz.

%%Unlike I/O clock used for clocking other Timers/Counters,
%%this asynchronous crystal is also active in power-save mode~\ref{avr-datasheet}.
%CITATION: If Timer/Counter2 is enabled, it will keep running during sleep. The device can wake up from
%either Timer Overflow or Output Compare event from Timer/Counter2.
%If Timer/Counter2 is not running, Power-down mode is recommended instead of Power-save
%mode.
%The Timer/Counter2 can be clocked both synchronously and asynchronously in Power-save
%mode. If the Timer/Counter2 is not using the asynchronous clock, the Timer/Counter Oscillator is
%stopped during sleep. If the Timer/Counter2 is not using the synchronous clock, the clock source
%is stopped during sleep. Note that even if the synchronous clock is running in Power-save, this
%clock is only available for the Timer/Counter2.

The Timer/Counter~2 module is used in Clear Timer on Compare Match (CTC) mode by Contiki.
In this mode, the counter register {\it{TCNT2}} is incrementing
and the compare register defines maximum value for the counter register.
Compare match between counter register and compare register
sets the Output Compare Flag {\it{OCF2A}} and resets the timer to zero~\cite{avr-datasheet}.
This behaviour illustrates figure~\ref{fig:design-timing-diagram}
- value {\it{TOP}} is equal to value in compare register and value {\it{BOTTOM}} is equal to zero.

\begin{figure}
  \centering
  \includegraphics[width=12cm,keepaspectratio]{fig/timing-diagram.pdf}
  \caption{Timing diagram in CTC mode with prescaler 8 (source:~\cite{avr-datasheet})}
  \label{fig:design-timing-diagram}
\end{figure}

Additionally, when compare match occurs,
interrupt is raised and interrupt service routine {\it{AVR\_OUTPUT\_COMPARE\_INT}},
defined in {\it{cpu/avr/dev/clock.c}} file, is executed.
In this case is flag indicating occurred match {\it{OCF2A}}
cleared automatically by hardware when executing
the interrupt service routine~\cite{avr-datasheet}.

\begin{figure}
  \centering
  \includegraphics[width=9cm,keepaspectratio]{fig/avr-clock.png}
  \caption{AVR Raven hardware clock}
  \label{fig:avr-clock}
\end{figure}

As described in section~\ref{sec:contiki-timers}, there is
{\it{CLOCK\_SECOND}} macro expressing number of clock interrupts per second.
To obtain {\it{CLOCK\_SECOND}} interrupts per second, there must be
${\frac{f_{T2}}{CLOCK\_SECOND}}$ hardware clock ticks between two successive interrupts.
On compare match in CTC mode, the timer is reset to zero as
shown in figure~\ref{fig:design-timing-diagram}.
The value zero is also included in the counting - the 0th count of the timer also takes one tick.
Therefore the value of compare register {\it{OCR2A}} must be ${\frac{f_{T2}}{CLOCK\_SECOND}} - 1$
when using Timer/Counter~2 in CTC mode.
Default value of {\it{CLOCK\_SECOND}} for AVR Raven in Contiki is 128,
defined in {\it{platform/avr-raven/contiki-conf.h}} file,
what implies default value of compare register ${\frac{4096}{128}} - 1 = 31$.

One of the tasks of the interrupt service routine is to increment the {\it{scount}} variable,
counting interrupts.
When this variable reaches the {\it{CLOCK\_SECOND}} value, the variable called
{\it{seconds}} is incremented and the {\it{scount}} variable is reset to zero.
The {\it{seconds}} variable is hence counting seconds since the system booted (uptime)
and is used by stimers for measuring time intervals.



\section{Communication}
Without any NTP server is an NTP client useless. % There are clients, servers...
However, too many server associations complicate the client design.
In fact, in the most common scenario, there can be only a single NTP master server
for the whole network.
A single server association requires just a simple calculation of the local clock offset
$\theta$, whereas more server associations require the intersection algorithm
described in section~\ref{sec:ntp-algorithms}.
Implementation of such an algorithm, requiring advanced data structures, should be avoided
in a memory constrained system.

The NTP broadcast communication mode, on the other hand,
requires no associations and no packet filling process on the client side.
Moreover, because the client does not have to actively send any NTP packets,
an energy consumption of the client is reduced.

% 1 - see design
A problem might be a possible packet loss when communication uses UDP on transport layer.
The reason why this can happen often in Contiki, is explained in section~\ref{sec:contiki-uip}.
% 2
In NTP unicast mode, the packet loss might occur either for a client's query to the server
or for a server's response to the client.
If the client's query loss occurs, no server response will be sent.
Similarly, if the server's response loss occurs, no message will be received by the client.
Not to block a whole system till the response arrives
is therefore a desired behaviour of the client.

The NTP client should be able to communicate over both IPv4 and IPv6.
Thanks to the uIP stack, this is no a matter for Contiki.
The only constraint is that both IP versions can not be used simultaneously
and the decision must be made during the compilation~\cite{contiki-docs}.
Although the {\it{UIP\_CONF\_IPV6}} macro can be used to determine which IP version
support is being compiled, the NTP client application can be written IP-version agnostic.


\section{Timestamp conversions}%CLIENT CODE
As mentioned in section~\ref{sec:ntp-time}, the NTP timescale is not
coincident with the POSIX timescale.
If the new call interface should use the standard POSIX timescale,
conversion between NTP and POSIX timestamps must be calculated.
The conversion from the POSIX timestamp to the 64-bit long NTP timestamp
is needed when the client sends the request
and the conversion vice versa is needed when the client calculates
the local clock offset from the received timestamps.

Since both timescales reckon in seconds, the conversion between
the NTP timestamp seconds field value and the POSIX timestamp seconds field value is simple.
However, the conversion between the NTP fraction field value ($2^{-32}$)
and the POSIX fraction field value (nanoseconds or microseconds) is problematic.
The relation between the POSIX fraction field and the NTP fraction field
is given by $POSIX.frac = NTP.frac \times POSIX.res \div 2^{32}$,
where $POSIX.frac$ is the POSIX fraction field value,
$NTP.frac$ is the NTP fraction field value and
$POSIX.res$ is the POSIX timestamp resolution (microseconds or nanoseconds).
The accurate conversion requires either floating point operations or operations with 64 bit numbers.
These operations can be memory expensive, especially on 8-bit microcontrollers,
and their usage must be considered carefully or another suitable solution must be found.


%=========================================================================
% (c) 2011, 2012 Josef Lusticky

\chapter{Design}
The analysis showed, that just an implementation of the
NTP client application is, due to missing calls in the time interface, not sufficient.
Analysis further described the necessary components of the NTP client,
as shown in figure~\ref{fig:analysis-overview}.

The uIP stack, described in section~\ref{sec:contiki-uip}, provides a feature-rich
communication interface for the NTP client application.
The communication interface is therefore not a matter for Contiki OS.
However, the time interface must be extended with new calls,
that must be further implemented in the clock library.
Figure~\ref{fig:design-overview} shows the overview of the NTP client running on AVR Raven.

\begin{figure}[H]
  \centering
  \includegraphics[width=13cm,keepaspectratio]{fig/design.png}
  \caption{NTP client overview} %!TODO
  \label{fig:design-overview} %!TODO
\end{figure}

%=========================================================================
% (c) 2011, 2012 Josef Lusticky

\section{Operating system time interface - TODO}
The main problem of NTP client implementation for Contiki is therefore a total
lack of real-time support.
Not only no common interface is available, but also
almost no platform-specific code has been implemented towards time interface yet.



-- git

Setting the current time is only possible within one second precision -
finer time setting must be made through time adjustments described further.
Implemented {\it{clock\_set\_time}} function computes when the system started
in seconds since the Epoch and saves the result in newly implemented {\it{boottime}} variable.



Unlike GIT!
This variable, counting uptime in seconds,
is particularly used by stimers in Contiki
and modifying it would lead to misbehaviour of stimer library
described in section~\ref{sec:contiki-timers}.


If the operating system implements the kernel discipline described in section~\ref{sec:system-discipline},
an NTP client will announce insertion and deletion of a leap second to the kernel and
the kernel will handle the leap second without further action necessary~\cite{ntp-faq}.
If the operating system does not implement the kernel discipline,
the system clock will show an error of one second after the leap second occurred~\cite{ntp-faq}.
The situation will be handled just like an unexpected change of time -
the operating system will continue with the wrong time for some time,
but an NTP client will step or adjust the time~\cite{ntp-faq}.
This will effectively cause the leap second correction to be applied too late.
% which is a trade-off for smaller memory requirements

Since there is no way of setting, getting and adjusting the time in Contiki OS,
a new interface for setting, getting and eventually
adjusting the time must be developed.


%=========================================================================
% (c) 2011, 2012 Josef Lusticky

\section{Clock interface}\label{sec:design-clock}
Previous section described how the call for getting the time acquires
the maximum precision the clock model allows.
In such a design, there are two read operations - read {\it{scount}} and read {\it{TCNT2}}.
Since the {\it{scount}} variable depends on asynchronous interrupts produced by
the clock module, the followed query of the counter register causes a race condition.
The timer clock runs asynchronously from the CPU clock and
the result may be unpredictable if read while the timer is running.
Although the read could be wrapped with an interrupt disable,
the common solution on AVR platform in Contiki is to perform more read operations,
compare the results and perform read operations again if the results are not consistent.
Figure~\ref{fig:design-read} illustrates such a solution.

\begin{figure}
  \centering
  \includegraphics[width=6cm,keepaspectratio]{fig/read.png}
  \caption{Multiple read and result comparison}
  \label{fig:design-read}
\end{figure}

The call for adjusting the time computes the number of clock ticks
with longer or shorter tick interval.
When adjusting the time, the %TODO
as follows:

CLOCK\_COMPARE\_REGISTER = 30 => ca132.129Hz => 1s = ca1.032258s
FREQ = 32768/8 / 31
CLOCK\_COMPARE\_REGISTER = 32 => 124.12per => 1s = 0.96p
FREQ = 32768/8 / 33

Adjusting time - CLOCK\_COMPARE\_REGISTER = 31 => 128Hz => 1s = 1s
FREQ = 32768/8 / 32

The fastest adjust is 0.03 $\frac{s}{s}$.

% TO NTP INTERFACE
When writing to compare register, the value is transferred to a
temporary register, and latched after two positive edges of a source clock~\cite{avr-datasheet}.
The user should not write a new value before the contents
of the temporary register have been transferred to its destination.
To detect that a transfer to the destination register has taken place,
the Asynchronous Status Register - ASSR has been implemented.
Since writing to compare register occurs only once within interrupt CONTEXT, % context?
this detection is not mandatory.



This is enough for implementing a reasonable time interface and using it for NTP client later.

% ntp interface extending the clock library, similar to posix calls


Each TCNT2 increment is $\frac{1}{128 \times 32} \doteq 0,000244$s
0,244ms
This is also minimal possible clock adjustment.


Please note, that these adjustments will influence Contiki timers.
Applications requiring uninfluenced timers
are therefore advised to use rtimers, described in section~\ref{sec:contiki-timers},
because they use separate hardware clock unaffected by NTP client
(Timer/Counter~3 on AVR Raven platform).


%%=========================================================================
% (c) 2011, 2012 Josef Lusticky

\section{Network communication}
Thanks to uIP, described in section~\ref{sec:contiki-uip},
the network communication is not a matter for Contiki OS.
% 1 - see implementation/communication.tex
A problem might be a possible packet loss when communication uses UDP on transport layer.
The reason why this can happen often is explained in section~\ref{sec:contiki-uip}.
% 2
In NTP unicast mode, the packet loss might occur either for client's query to server
or for server's response to client.
If the client's query loss occurs, no server response will be sent.
Similarly, if the server's response loss occurs, no message will be received by the client.
Not to block a whole system till the response arrives
is therefore a desired behaviour of the client.

--- to design.tex ?
The AVR Raven platform features 802.15.4 support.
Contiki is used in conjunction with RZ USB Stick - 6LoWPAN.

The 6LoWPAN Adaptation Layer % see Interconnecting smart objects with IP
% IEEE 802.15.4
as RFC~4944 written by 6lowpan working group of IETF
made the underlaying IEEE 802.15.4 layer
look like an IPv6 link~\cite{6lowpan} and
\begin{figure}
  \centering
  \includegraphics[width=9cm,keepaspectratio]{fig/6lowpan.pdf}
  \caption{Communication stack with 6lowpan layer}
  \label{fig:design-6lowpan}
  \bigskip
\end{figure}


Contiki supports broadcast packets as well as sending multicast packets.
Joining multicast groups through Internet Group Management Protocol (IGMP)
and receiving non-local multicast packets
was not supported at the time of writing~\cite{contiki-docs}.
Contiki is also able to use Domain Name System for IPv4 address resolution.
DNS resolution of IPv6 addresses was not implemented in Contiki OS
at the time of writing~\cite{contiki-docs}.


%=========================================================================
% (c) 2011, 2012 Josef Lusticky

\section{Contiki NTP client}\label{sec:design-client}
The client application itself is a Contiki process,
which will use the designed operating system interface from the previous sections
and the uIP communication stack.

The client should be able to use both NTP communication modes,
the NTP broadcast mode and the NTP unicast mode.
The NTP broadcast communication mode is intended particularly for energy or
even more memory constrained clients.
If the client will use only the broadcast mode, the structures and code
related to the unicast mode should not be included in the resulting program.

%TODO

As discussed in section~\ref{sec:analysis-application}, the NTP broadcast mode requires no server
associations.
A simple packet receiving is not a matter for constrained systems.
 %% CHECK !! The NTP broadcast mode packet can be received and processed from any NTP server in the network.


In NTP unicast mode, if the NTP client support towards more NTP servers,
the intersection algorithm, described in section~\ref{sec:ntp-algorithms}, would be needed.

In NTP unicast mode, the server associations are needed in the client.
This would complicate the client design and 
Support for more servers would also need the algorithms described in section~\ref{sec:ntp-algorithms}.

Section described, a network with
and the algorithm is complicated.
A single NTP server is the most common use case for a Simple Network Time Protocol client.

A simple NTP client (SNTP client) has a single NTP server~\cite{rfc5905}.


but it can send the request to only one specified NTP server.




%TODO - Future work:
%It is useful to provide an initial volley where the client operating in
%client mode exchanges several packets with the server, so as to
%calibrate the propagation delay and to run the Autokey security
%protocol, after which the client reverts to broadcast client mode~\cite{rfc5905}.



The NTP client fills and checks only the seconds part of the NTP timestamp,
because the conversion to the NTP format would increase the interval
between the timestamp determination and the dispatch of the filled packet.

After the filled NTP packet is sent, the client schedules
the dispatch of the next NTP packet in $2^{\tau}$ seconds
using the event timer library.
In NTPv4, $\tau$ ranges from 4, resulting in NTP poll interval of 16 seconds,
to 17, resulting in NTP poll interval of 36 hours.
However, the event timer library imposes a limit on the scheduled time.
This limit is platform specific and depends on the {\it{CLOCK\_SECOND}} value,
e.g. the $\tau$ value can not be greater than 8 on AVR Raven assuming 128 interrupts per second.
Upon scheduling the event timer, the client process yields
and another process can be run.
The client process is later invoked either by the uIP stack event
announcing the server response
or by the event timer in case no server response arrived.
The event timer therefore effectively solves
the possible packet loss problem described in section~\ref{sec:analysis-application}.


%%
The packet loss problem was described in section~\ref{sec:analysis-application}.
However, packet loss is not a matter for NTP if using either broadcast or unicast mode.
In broadcast mode, a lost server packet causes no setting or adjusting the client's system time.
The client simply waits without disruption for the next NTP broadcast message.
If the client needs to figure out it's local clock offset at the moment,
it can simply query the server using the NTP unicast mode.

%%%TODO


When the server response arrives,
the destination timestamp determination is one of the first tasks the client does.
After that, the client makes packet sanity tests, including
checking whether the response is from the synchronised server.

A determination of the NTP communication mode follows.
In the unicast mode, the Originate timestamp is compared with the stored sent timestamp.

The received packet is considered bogus in case of mismatch and further processing is stopped.
Otherwise, the NTP timestamps are converted to the local timestamp format and
the local clock offset is computed as described in section~\ref{sec:ntp-algorithms}.
After the local clock offset is computed,
the stored transmitted timestamp is immediately set to zero
to protect against a replay of the last transmitted packet.

In broadcast mode, the received packet is always considered correct
and the local clock offset is computed as the difference between the local stored timestamp
and the received Transmit timestamp.
The local clock offset determined from the broadcast mode
is influenced by the network propagation delay and therefore less accurate.

The NTP client could exchange several packets with the server to calibrate the propagation delay.
But since local variables can not be reused in the Contiki process when the process yields,
this would cause either an extra memory overhead or a complicate client design.





%DONE
Section~\ref{sec:design-interface} described that the client uses the POSIX timescale,
whereas NTP uses the NTP timescale.
Because the time is reckoned in seconds by both timescales,
the number of seconds between the NTP epoch and the POSIX epoch
can be simply subtracted from the seconds part of the NTP timestamp.
But the conversion from the fraction part of the long 64-bit NTP timestamp to nanoseconds,
used in the local timestamp structure,
is one of the most problematic tasks for memory constrained systems.
An accurate conversion requires either floating point operations or operations including 64-bit numbers~\cite{c99}.
The conversion is given by
$nsec = fractionl \times 10^9 \div 2^{32}$, where $nsec$ is the nanoseconds part of the local timestamp
and $fractionl$ is the fraction part of the long 64-bit NTP timestamp.

Since there is no native hardware support for floating point nor 64-bit arithmetic operations,
the compiler supplies these operations in form of library, called {\it{libgcc}} in case of the GCC compiler,
which causes a significantly bigger resulting binary file
(kilobytes in case of floating point operations and hundreds of bytes in case of 64-bit operations).
The greatest common divisor of $10^9$ and $2^{32}$ is $2^9$,
so in fact, a relatively simple multiplication of $fractionl$ by $\frac{5^9}{2^{23}}$ must be computed.
This could be computed on 32 bits using sequential
divisions by the power of 2 and multiplications by the power of 5.
In the standard C programming language, the bitwise right shift operator divides the unsigned data type by the power of 2
and the bitwise left shift operator multiplies the unsigned data type by the power of 2~\cite{c99}.
Therefore, the multiplication by 5 can be done using two left shifts and
adding the original value ($5x = 4x + x$).
The only constraint is that the overall coefficient of these operations must not be greater than 1,
that is, the value must be in the range from $0$ to $2^{32}-1$ in every step.
Otherwise, the value could overflow and the result would be incorrect.
Division can not cause such a situation but multiplication could.
The original value could be divided by a greater divisor,
but this would lead to a greater inaccuracy due to the lost of the least significant bits.
Because of this, multiplication done as soon as possible provides more accurate results.
The ideal conversion sequence is therefore given by formula~\ref{equ:conversion}.
\begin{equation}
\label{equ:conversion}
%\frac{10^9}{2^{32}} = \frac{2^9 \times 5^9}{2^9 \times 2^{23}} =
%\frac{5^9}{2^{23}} =
nsec = fractionl \times \frac{5}{2^3} \times \frac{5}{2^3} \times \frac{5}{2^3} \times \frac{5^2}{2^3} \times \frac{5}{2^3}  \times \frac{5}{2^3} \times \frac{5^2}{2^3} \times \frac{1}{2^2}
\end{equation}
It must be noted, that the above presented conversion is not exactly accurate,
because the least significant bits are lost by right shifts. %!LYDIA by
The accuracy can be determined by looping through all the possible values of $fractionl$ %!LYDIA by
and comparing the results with the reference algorithm that uses the floating point operations.
Such a measurement reports the maximum error of 5 nanoseconds,
which is totally adequate for most platforms without the floating point unit or
for platforms where 64-bit multiplication is expensive.
The implementation of the above as well as the program used for the
error measurement can be found on the CD attached to this thesis.
The table of CD contents is listed in appendix~\ref{app:cd-contents}.


After the timestamps were converted, the local clock offset is computed
as given in section~\ref{sec:ntp-algorithms}.
Depending on the absolute value of the local clock offset,
the system time is either set or adjusted using the {\it{clock\_set\_time}}
or {\it{clock\_adjust\_time}} call, respectively.
The clock is set if the time difference is equal or greater than
the offset threshold value.
The NTP specification suggests 0.125~seconds as the default~\cite{rfc5905}.
Because the designed call for setting the time, described in section~\ref{sec:design-interface},
can set the time only within a resolution of one second,
the threshold value must be at least one second.



%=========================================================================
% (c) 2011, 2012 Josef Lusticky

\chapter{NTP client in Contiki OS}

%... This can be effectively solved by NTP Poll Interval.

%=========================================================================
% (c) 2011, 2012 Josef Lusticky

\section{Clock library extension}
%!TODO
The {\it{clock\_init}} call was extended with macros...

So in fact, the code presented in the section~\ref{}
is not a pseudocode, instead it is the real code used.

This makes portability simple.... porting the code to a new platform simple.


%%CLOCK LIBRARY EXTENSION
Since the counter register can be of a different name on another AVR CPU
and the clock interface is common for all AVR CPUs,
a new general name {\it{CLOCK\_COMPARE\_REGISTER}} was defined in the {\it{clock\_init}} setup code
for the compare register {\it{OCR2A}}.
Similarly, the {\it{CLOCK\_COMPARE\_REGISTER}} was defined for the counter register {\it{TCNT2}},
the default value of the clock compare register was defined as {\it{CLOCK\_COMPARE\_DEFAULT\_VALUE}}
and the {\it{CLOCK\_CTC\_MODE}} was defined as 1, since the hardware clock is used in CTC mode,
which adds one counter register increment as described in section~\ref{sec:analysis-interface}.


%=========================================================================
% (c) 2011, 2012 Josef Lusticky


Getting the correct current time is only possible if it was set using
the {\it{clock\_set\_time}} function before.
Newly implemented function {\it{clock\_get\_time}} is then able to tell the
current time in seconds since the Epoch by simply adding {\it{boottime}},
and {\it{seconds}}.
%! TODO
Nanoseconds part is filled using {\it{scount}} variable counting number of
interrupts within a second.
Since this variable is incremented every interrupt and there are {\it{CLOCK\_SECOND}} interrupts
per second, it is possible to get resolution of $\frac{1~000~000~000}{CLOCK\_SECOND}$ ns.
The same resolution have Contiki timers, described in section~\ref{sec:contiki-timers}.
\begin{lstlisting}
void
clock_get_time(struct time_spec *ts)
{
  ts->sec = boottime + seconds;
  ts->nsec = (1000000000 / CLOCK_SECOND) * scount;
}
\end{lstlisting}
Because {\it{1000000000}} and {\it{CLOCK\_SECOND}} are both constants, the compiler is able to
calculate the result of division during compile time.
Furthermore as both numbers are integers, the result is integer as well~\cite{c99}.
The most of CPU time is therefore spent on multiplication.
E.g. if the code is compiled using GCC version 4.3.5,
multiplication of two 32-bit variables takes 33 instructions including {\it{call}} and {\it{ret}}
instructions for entering and returning from the {\it{\_\_mulsi3}} routine, which computes
the result of multiplication.
%avr-objdump
According to AVR Instruction Set manual~\cite{avr-instruction-set},
this results in 48 clock cycles overhead,
which takes 3~000 nanoseconds assuming 16MHz CPU clock.
The timestamp provided is therefore not exact.
However, since this consumed time strongly depends on architecture and compiler specifications,
no correction was implemented to remove this inaccuracy.
The application must be instead aware that the timestamp is not exactly accurate.

TODO: Greater precision is further implemented by reading counter register.

TODO: Adjust time
POSIX:
Time values that are between two consecutive non-negative integer multiples
of the resolution of the specified clock are truncated down to the smaller multiple of the resolution.


%=========================================================================
% (c) 2011, 2012 Josef Lusticky <xlusti00@stud.fit.vutbr.cz>

\section{NTP client implementation}
Structures representing NTP message were borrowed from OpenNTPD NTP Unix daemon.
%They are not using the GCC extension for representing a bit field.

Packet sanity tests~\cite{ntp-arch}.

A client sends messages to each server with a poll interval of $2^{\tau}$
seconds, as determined by the poll exponent $\tau$ (tau).
In NTPv4, $\tau$ ranges from 4 (16 s) to 17 (36 h).


% filling the packet
Precision express strictly speaking elapsed time to read the system clock from userland~\cite{ntp-arch}.
However most implementation supply clock precision.
%Dragonfly BSD:
%wmsg.precision = -6;
%Chrony, NTP.org - getting resolution, gettimeofday, clock_getres
\begin{lstlisting}
// set clock precision - convert Hz to log2 - borrowed from OpenNTPD
int b = CLOCK_SECOND; // CLOCK_SECOND * OCR2A
int a;
for (a = 0; b > 1; a--, b >>= 1)
  {}
msg.precision = a;
\end{lstlisting}
This will work for clock precision greater or equal 1s, i.e. CLOCK\_SECOND must be greater or equal 1.
%refer to CLOCK\_SECOND - is always greater or equal 1

The clock are set if the time difference is greater than XX seconds. %! TODO
\begin{lstlisting}
if (labs((signed long) (ts.sec - tmpts.sec)) > 2)
{
  clock_set_time(ts.sec);
}
\end{lstlisting}
Even if {\it{tmpts.sec}} value is greater than {\it{ts.sec}} value,
subtracting and casting to signed type gives correct (negative) result~\cite{c99}.
Assuming 32-bit data types this will work until 2038 when wrap around can occur due to difference
between {\it{ts.sec}} and {\it{tmpts.sec}} greater than $2^{31}$-1 (2~147~483~647).
But as NTP Era 0 ends 2036 the NTP client code must be changed in the future anyway.

%! TODO

%Adjusting time
%1/128/32 = 0.000244141
%0.000244141x32x127+0.000244141x31 == smallest possible adjustment == 244us

%\section{NTP values and convertions}
Unlike the RFC 5905~\cite{rfc5905} shows, there are no 64 bit values. %! RFC - A.4. Kernel System Clock Interface
No floating point numbers - library.
Division of unsigned integer number by 2 is automatically translated by compiler to logical right shift,
making it fast operation.

Converting between NTP and local timestamps requires floating point operations or 64-bit numbers.
According to output from avr-size tool, using 64-int number for conversion
uses 4~330 bytes more in program section of resulted binary file 
and floating point operation takes 3~474 bytes more
than algorithm developed?

C99 - shift E1 >> E2: if E1 has a signed type and a nonnegative value, the value of
the result is the integral part of the quotient of $E1 / 2^{E2}$.

In case of floating point operations, the libgcc is used.
\url{http://gcc.gnu.org/onlinedocs/gccint/Libgcc.html}

%
% NEGATIVE result for the first time
In some scenarios where the initial frequency offset of the client is
  relatively large and the actual propagation time small, it is
   possible for the delay computation to become negative.  For instance,
   if the frequency difference is 100 ppm and the interval T4-T1 is 64
   s, the apparent delay is -6.4 ms.  Since negative values are
   misleading in subsequent computations, the value of delta should be
   clamped not less than s.rho, where s.rho is the system precision
   described in Section 11.1, expressed in seconds~\cite{rfc5905}.
%


%=========================================================================
% (c) 2011, 2012 Josef Lusticky

\section{Network communication}
% 1 - see design/network.tex
Communication over IEEE~802.15.4 link layer uses a 6LoWPAN adaptation layer.
Thanks to this layer, AVR Raven running Contiki OS is connected to the IPv6 Internet.
The client and the developed interface uses no IP version specific code,
therefore a communication over IPv4 should be also possible, though not tested.

The packet loss problem was described in section~\ref{sec:design-network}.
However, packet loss is not a matter for NTP if using either broadcast or unicast mode.
In broadcast mode, lost server packet causes no setting or adjusting time of client's
local clock.
The client simply waits without disruption for next NTP broadcast message.
If client needs to figure out it's local clock offset at the moment,
it can simply query a server using NTP unicast mode.
% 2
Upon sending the packet, the NTP client process yields
using the {\it{PROCESS\_YIELD}} statement, so no active waiting
causes blocking the whole system.

The remote NTP server can be specified in Makefile or
using the {\it{REMOTE\_HOST}} define macro.
If no remote host is specified,
NTP client assumes only NTP broadcast communication mode will be used.
The broadcast mode is intended particularly for energy or memory constrained clients
or for a huge number of NTP clients and a single NTP server
in a network with small propagation delay.
Should there be more devices running Contiki present in one network,
each of them needs a different link layer address.
This address can be configured in Makefile as well.
Beware that although the firmware for RZ~USB Stick automatically translates
between Ethernet link layer addresses and IEEE~802.15.4 link layer
addresses, both are are still different links and can not be mixed in one
layer~2 network (e.g. bridging will not work).

Dynamic increasing or decreasing the client's poll interval in response to
Kiss-o'-Death packets, described in section~\ref{sec:ntp-network}, is not implemented.
The configuration instead assumes, that an exhausted NTP server rather drops the incoming
client's packet than sending the response with KoD code.

Contiki NTP Client is primarily intended for use in local networks with a single master NTP server,
although using any NTP server found in the Internet would work.
Figure~\ref{fig:implementation-routing} shows the network topology used
for tests and measurements of Contiki NTP Client.
% How to set up the illustrated network is described in tutorial on CD.
The Meinberg primary NTP server was synchronised with PPS %todo.
Measurements made using this setup are discussed in chapter~\ref{chap:measurements}.

\begin{figure}
	\centering
	\includegraphics[width=10cm,keepaspectratio]{fig/radvd-routing.png}
	\caption{Contiki NTP Client communicating with remote NTP server}
	\label{fig:implementation-routing}
	%\bigskip
\end{figure}



%=========================================================================
% (c) 2011, 2012 Josef Lusticky

\chapter{Measurements}\label{chap:measurements}
There are several factors that can be measured.
The clock interrupt frequency measurements show the influence of clock adjustments
on the number of clock ticks (interrupts) per second.
The clock offset measurements show the time difference between the reference clock and
the local clock.
The clock phase measurements show the phase difference between the reference clock and
the local clock, that is, when each second is accounted.
\begin{figure}[H]
	\centering
	\includegraphics[width=13cm,keepaspectratio]{fig/radvd-routing.png}
	\caption{Contiki NTP Client communicating with the Meinberg NTP server}
	\label{fig:measurements-routing}
\end{figure}
Figure~\ref{fig:measurements-routing} shows the network topology used for the measurements.
The Meinberg~M600 NTP stratum 1 server synchronised using the GPS
was used as the reference clock for all of the presented measurements.
The NTP client was running on the AVR Raven platform in a room at 23~\textcelsius.

\section{Clock interrupt frequency}
For measuring clock interrupt frequency the bit 7 of Port D
and ground pin was connected to UNI-T~2025CEL digital oscilloscope.
At the beginning of the interrupt service routine a~logic 1 is written,
what causes a high level of voltage.
At the end of the interrupt service routine a~logic 0 is written,
what causes a low level of voltage.

When there is no clock adjustment, the value in output compare register is 31 by default.
The clock interrupt frequency
is supposed to be equal to a~value of the {\it{CLOCK\_SECOND}} macro, which is 128 by default on AVR~Raven.
The~figure~\ref{fig:app-osc-no-adjust} shows the output from oscilloscope
for this case.
$$\frac{\frac{f_{asy}}{prescaler}}{counts} = \frac{\frac{32768}{8}}{32} = 128$$

Figure~\ref{fig:app-osc-speed-up} shows the~output from oscilloscope
when slowing down the clock.
The~clock interrupt frequency
is supposed to be equal to $124.\overline{12}$~Hz.
$$\frac{\frac{f_{asy}}{prescaler}}{counts + 1} = \frac{\frac{32768}{8}}{32+1} = 124.\overline{12}$$

Figure~\ref{fig:app-osc-slow-down} shows the output from oscilloscope
when speeding up the clock.
The~clock interrupt frequency
is supposed to be approximately equal to 132.129~Hz.
$$\frac{\frac{f_{asy}}{prescaler}}{counts - 1} = \frac{\frac{32768}{8}}{32-1} \doteq 132.129$$

The~measured values are not exactly equal to those expected.
This is mostly due to influence of the clock source
(32~768~Hz quartz crystal oscillator) by a room temperature,
but it could also be air pressure or magnetic fields, etc.

\section{Clock offset}
\begin{figure}[H]
  \centering
  \includegraphics[width=11cm,keepaspectratio]{fig/no-ntp.png}
  \caption{Local clock offset without NTP client}
  \label{fig:measurements-no-ntp}
\end{figure}
Figure~\ref{fig:measurements-no-ntp} shows the local clock offset
in case no NTP client runs on the device.
The time is set with the initial offset of about 45 milliseconds.
However, the clock progresses faster due to frequency errors.
This compensates the initial clock offset at first,
but then it causes the offset increase.
The clock is running faster with approximately 100~PPM
(9 seconds a day) in this case and
the frequency jitter can be also observed.
The offset increase is therefore not exactly linear.

Figure~\ref{fig:measurements-ntp-serial} shows the local clock offset
acquired from the serial output when Contiki NTP Client runs on the device.
When the developed NTP client receives response from the server,
it calculates the local clock offset and prints this value to the serial output.
The NTP poll interval was set to 16 seconds, that means, the local clock offset
is calculated and eventually corrected every 16 seconds.
\begin{figure}[H]
  \centering
  \includegraphics[width=11cm,keepaspectratio]{fig/poll-16s.png}
  \caption{Local clock offset with adjustments and NTP poll interval 16s}
  \label{fig:measurements-ntp-serial}
\end{figure}
The blue line shows the mean local clock offset value,
that should be equal to zero in a perfect case.
This is however not the case, because of oscillator frequency error
shown in figure~\ref{fig:measurements-no-ntp}.
More figures showing the local clock offset measurement
can be found in appendix~\ref{app:offset}.

\section{Clock phase}
The GPS based clock Meinberg~GPS~167 and digital oscilloscope UNI-T~2025CEL
was used for measuring the clock phase difference.
Meinberg~GPS clock rises an impulse when each second is accounted.
Contiki on AVR~Raven was configured to write a logic~1
to~bit~7 of~Port~D when each second is accounted,
and to write a logic~0 to~the same bit after~25 clock ticks.

When NTP client uses the {\it{clock\_adjust\_time}} call,
the local clock offset as well as the phase is being adjusted.
Figure~\ref{fig:measurements-osc-adjusting-phase} shows the phase while adjusting the clock.
The yellow line is the output signal from Meinberg~GPS clock
and the blue line is the output signal from AVR~Raven.
\begin{figure}[H]
  \centering
  \includegraphics[width=11cm,keepaspectratio]{fig/osc-adjusting-phase.png}
  \caption{Second impulses when the clock is being adjusted}
  \label{fig:measurements-osc-adjusting-phase}
\end{figure}
Figures showing the clock out of phase and the clock in phase with
the reference clock can be found in appendix~\ref{app:phase}.


%! finish
\chapter{Conclusion}
The Network Time Protocol has been operating over 30 years as of early 2012
and remains the longest running, continuously operating application
protocol in the Internet~\cite{ntp-y2k}.

The developed NTP client for Contiki OS is able to use the NTP unicast and broadcast mode.
The implemented time interface extends Contiki OS toward real-time support,
while requiring minimal memory amounts.
The unique timestamp conversion provides a perfect solution to avoid the floating point or 64-bit
arithmetic operations.
The measured results show that the Contiki NTP client is satisfying for keeping a reasonably accurate time.

The Network Time Protocol is suitable for constrained devices
and will arguably find its place among embedded systems in the near future.

%The implemented clock functionality




TODO - Future work:
It is useful to provide an initial volley where the client operating in
client mode exchanges several packets with the server, so as to
calibrate the propagation delay and to run the Autokey security
protocol, after which the client reverts to broadcast client mode~\cite{rfc5905}.

Refid~\cite{rfc5905}?? - not necessary

Ability to communicate with more servers. Requires clock selection algorithms.

Advanced clock discipline algorithms -
The clock discipline process is a system process that controls the
time and frequency of the system clock~\cite{rfc5905},

%Add timestamp into i-node in CFS.

%--
%ntpv4.pdf:
%SNTP is intended for primary
%servers equipped with a single reference clock, as well as clients with a single upstream server
%and no dependent clients.

%It is useful to provide an
%initial volley where the client operating in mode 3 exchanges several packets with the server in
%order to calibrate the propagation delay % ntp/algorithms.tex

%The operating system is assumed to provide two functions, one to set the time
%directly, for example the Unix settimeofday()1 function, and another to adjust the time in small
%increments advancing or retarding the time by a designated amount, for example the Unix
%adjtime() function. In the intended design the clock discipline process uses the adjtime() function
%if the adjustment is less than a designated threshold, and the settimeofday() function if above the
%threshold.

%An SNTP client using the on-wire protocol has a single server and no downstream clients. It can
%operate with any subset of the NTP on-wire protocol, the simplest using only the transmit
%timestamp of the server packet and ignoring all other fields. However, the additional complexity
%to implement the full on-wire protocol is minimal and is encouraged.


%--
%rfc5905:
%In the case of NTP as specified herein, NTP broadcast clients are
%vulnerable to disruption by misbehaving or hostile SNTP or NTP
%broadcast servers elsewhere in the Internet.  Such disruption can be
%minimized by several approaches.  Filtering can be employed to limit
%the access of NTP clients to known or trusted NTP broadcast servers.
%Such filtering will prevent malicious traffic from reaching the NTP
%clients.
