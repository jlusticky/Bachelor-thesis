%=========================================================================
% (c) 2011, 2012 Josef Lusticky

\chapter{Design and analysis}
For implementation of reasonably useful NTP client,
operating system must be able to set, get
and eventually adjust the system time in a similar fashion as
described in section~\ref{sec:system-keeping-and-providing} and~\ref{sec:system-discipline}.
Though not mandatory, adjusting time is important function,
if the time shall be always a monotonically increasing function.
Apart from that, communicating abilities over UDP are also required.

For developing and testing Contiki NTP client,
the AVR Raven platform with 8-bit ATmega1284P CPU~\cite{avr-datasheet} will be used.
This platform features IEEE~802.15.4 (Low-Rate Wireless Personal Area Networks) link layer support.
Together with an adaptation layer called 6LoWPAN (IPv6 over Low power Wireless Personal Area Networks)
AVR Raven is able to communicate over IPv6.

How to get a working setup with Contiki on this platform is described in
the document files on the CD enclosed to this thesis.
The CD content hierarchy is listed in appendix~\ref{app:cd-content}.

%... This is not in Contiki, as operating systems targeted at embbeded systems produce only 1 binary file. CITATION

%=========================================================================
% (c) 2011, 2012 Josef Lusticky

\section{Network communication}
Thanks to uIP, described in section~\ref{sec:contiki-uip},
the network communication is not a matter for Contiki OS.
% 1 - see implementation/communication.tex
A problem might be a possible packet loss when communication uses UDP on transport layer.
The reason why this can happen often is explained in section~\ref{sec:contiki-uip}.
% 2
In NTP unicast mode, the packet loss might occur either for client's query to server
or for server's response to client.
If the client's query loss occurs, no server response will be sent.
Similarly, if the server's response loss occurs, no message will be received by the client.
Not to block a whole system till the response arrives
is therefore a desired behaviour of the client.

--- to design.tex ?
The AVR Raven platform features 802.15.4 support.
Contiki is used in conjunction with RZ USB Stick - 6LoWPAN.

The 6LoWPAN Adaptation Layer % see Interconnecting smart objects with IP
% IEEE 802.15.4
as RFC~4944 written by 6lowpan working group of IETF
made the underlaying IEEE 802.15.4 layer
look like an IPv6 link~\cite{6lowpan} and
\begin{figure}
  \centering
  \includegraphics[width=9cm,keepaspectratio]{fig/6lowpan.pdf}
  \caption{Communication stack with 6lowpan layer}
  \label{fig:design-6lowpan}
  \bigskip
\end{figure}


Contiki supports broadcast packets as well as sending multicast packets.
Joining multicast groups through Internet Group Management Protocol (IGMP)
and receiving non-local multicast packets
was not supported at the time of writing~\cite{contiki-docs}.
Contiki is also able to use Domain Name System for IPv4 address resolution.
DNS resolution of IPv6 addresses was not implemented in Contiki OS
at the time of writing~\cite{contiki-docs}.


%=========================================================================
% (c) 2011, 2012 Josef Lusticky

\section{Clock interface}\label{sec:design-clock}
Previous section described how the call for getting the time acquires
the maximum precision the clock model allows.
In such a design, there are two read operations - read {\it{scount}} and read {\it{TCNT2}}.
Since the {\it{scount}} variable depends on asynchronous interrupts produced by
the clock module, the followed query of the counter register causes a race condition.
The timer clock runs asynchronously from the CPU clock and
the result may be unpredictable if read while the timer is running.
Although the read could be wrapped with an interrupt disable,
the common solution on AVR platform in Contiki is to perform more read operations,
compare the results and perform read operations again if the results are not consistent.
Figure~\ref{fig:design-read} illustrates such a solution.

\begin{figure}
  \centering
  \includegraphics[width=6cm,keepaspectratio]{fig/read.png}
  \caption{Multiple read and result comparison}
  \label{fig:design-read}
\end{figure}

The call for adjusting the time computes the number of clock ticks
with longer or shorter tick interval.
When adjusting the time, the %TODO
as follows:

CLOCK\_COMPARE\_REGISTER = 30 => ca132.129Hz => 1s = ca1.032258s
FREQ = 32768/8 / 31
CLOCK\_COMPARE\_REGISTER = 32 => 124.12per => 1s = 0.96p
FREQ = 32768/8 / 33

Adjusting time - CLOCK\_COMPARE\_REGISTER = 31 => 128Hz => 1s = 1s
FREQ = 32768/8 / 32

The fastest adjust is 0.03 $\frac{s}{s}$.

% TO NTP INTERFACE
When writing to compare register, the value is transferred to a
temporary register, and latched after two positive edges of a source clock~\cite{avr-datasheet}.
The user should not write a new value before the contents
of the temporary register have been transferred to its destination.
To detect that a transfer to the destination register has taken place,
the Asynchronous Status Register - ASSR has been implemented.
Since writing to compare register occurs only once within interrupt CONTEXT, % context?
this detection is not mandatory.



This is enough for implementing a reasonable time interface and using it for NTP client later.

% ntp interface extending the clock library, similar to posix calls


Each TCNT2 increment is $\frac{1}{128 \times 32} \doteq 0,000244$s
0,244ms
This is also minimal possible clock adjustment.


Please note, that these adjustments will influence Contiki timers.
Applications requiring uninfluenced timers
are therefore advised to use rtimers, described in section~\ref{sec:contiki-timers},
because they use separate hardware clock unaffected by NTP client
(Timer/Counter~3 on AVR Raven platform).


%=========================================================================
% (c) 2011, 2012 Josef Lusticky

\section{Operating system time interface - TODO}
The main problem of NTP client implementation for Contiki is therefore a total
lack of real-time support.
Not only no common interface is available, but also
almost no platform-specific code has been implemented towards time interface yet.



-- git

Setting the current time is only possible within one second precision -
finer time setting must be made through time adjustments described further.
Implemented {\it{clock\_set\_time}} function computes when the system started
in seconds since the Epoch and saves the result in newly implemented {\it{boottime}} variable.



Unlike GIT!
This variable, counting uptime in seconds,
is particularly used by stimers in Contiki
and modifying it would lead to misbehaviour of stimer library
described in section~\ref{sec:contiki-timers}.


If the operating system implements the kernel discipline described in section~\ref{sec:system-discipline},
an NTP client will announce insertion and deletion of a leap second to the kernel and
the kernel will handle the leap second without further action necessary~\cite{ntp-faq}.
If the operating system does not implement the kernel discipline,
the system clock will show an error of one second after the leap second occurred~\cite{ntp-faq}.
The situation will be handled just like an unexpected change of time -
the operating system will continue with the wrong time for some time,
but an NTP client will step or adjust the time~\cite{ntp-faq}.
This will effectively cause the leap second correction to be applied too late.
% which is a trade-off for smaller memory requirements

Since there is no way of setting, getting and adjusting the time in Contiki OS,
a new interface for setting, getting and eventually
adjusting the time must be developed.


%%=========================================================================
% (c) 2011, 2012 Josef Lusticky

\section{Contiki NTP client}\label{sec:design-client}
The client application itself is a Contiki process,
which will use the designed operating system interface from the previous sections
and the uIP communication stack.

The client should be able to use both NTP communication modes,
the NTP broadcast mode and the NTP unicast mode.
The NTP broadcast communication mode is intended particularly for energy or
even more memory constrained clients.
If the client will use only the broadcast mode, the structures and code
related to the unicast mode should not be included in the resulting program.

%TODO

As discussed in section~\ref{sec:analysis-application}, the NTP broadcast mode requires no server
associations.
A simple packet receiving is not a matter for constrained systems.
 %% CHECK !! The NTP broadcast mode packet can be received and processed from any NTP server in the network.


In NTP unicast mode, if the NTP client support towards more NTP servers,
the intersection algorithm, described in section~\ref{sec:ntp-algorithms}, would be needed.

In NTP unicast mode, the server associations are needed in the client.
This would complicate the client design and 
Support for more servers would also need the algorithms described in section~\ref{sec:ntp-algorithms}.

Section described, a network with
and the algorithm is complicated.
A single NTP server is the most common use case for a Simple Network Time Protocol client.

A simple NTP client (SNTP client) has a single NTP server~\cite{rfc5905}.


but it can send the request to only one specified NTP server.




%TODO - Future work:
%It is useful to provide an initial volley where the client operating in
%client mode exchanges several packets with the server, so as to
%calibrate the propagation delay and to run the Autokey security
%protocol, after which the client reverts to broadcast client mode~\cite{rfc5905}.



The NTP client fills and checks only the seconds part of the NTP timestamp,
because the conversion to the NTP format would increase the interval
between the timestamp determination and the dispatch of the filled packet.

After the filled NTP packet is sent, the client schedules
the dispatch of the next NTP packet in $2^{\tau}$ seconds
using the event timer library.
In NTPv4, $\tau$ ranges from 4, resulting in NTP poll interval of 16 seconds,
to 17, resulting in NTP poll interval of 36 hours.
However, the event timer library imposes a limit on the scheduled time.
This limit is platform specific and depends on the {\it{CLOCK\_SECOND}} value,
e.g. the $\tau$ value can not be greater than 8 on AVR Raven assuming 128 interrupts per second.
Upon scheduling the event timer, the client process yields
and another process can be run.
The client process is later invoked either by the uIP stack event
announcing the server response
or by the event timer in case no server response arrived.
The event timer therefore effectively solves
the possible packet loss problem described in section~\ref{sec:analysis-application}.


%%
The packet loss problem was described in section~\ref{sec:analysis-application}.
However, packet loss is not a matter for NTP if using either broadcast or unicast mode.
In broadcast mode, a lost server packet causes no setting or adjusting the client's system time.
The client simply waits without disruption for the next NTP broadcast message.
If the client needs to figure out it's local clock offset at the moment,
it can simply query the server using the NTP unicast mode.

%%%TODO


When the server response arrives,
the destination timestamp determination is one of the first tasks the client does.
After that, the client makes packet sanity tests, including
checking whether the response is from the synchronised server.

A determination of the NTP communication mode follows.
In the unicast mode, the Originate timestamp is compared with the stored sent timestamp.

The received packet is considered bogus in case of mismatch and further processing is stopped.
Otherwise, the NTP timestamps are converted to the local timestamp format and
the local clock offset is computed as described in section~\ref{sec:ntp-algorithms}.
After the local clock offset is computed,
the stored transmitted timestamp is immediately set to zero
to protect against a replay of the last transmitted packet.

In broadcast mode, the received packet is always considered correct
and the local clock offset is computed as the difference between the local stored timestamp
and the received Transmit timestamp.
The local clock offset determined from the broadcast mode
is influenced by the network propagation delay and therefore less accurate.

The NTP client could exchange several packets with the server to calibrate the propagation delay.
But since local variables can not be reused in the Contiki process when the process yields,
this would cause either an extra memory overhead or a complicate client design.





%DONE
Section~\ref{sec:design-interface} described that the client uses the POSIX timescale,
whereas NTP uses the NTP timescale.
Because the time is reckoned in seconds by both timescales,
the number of seconds between the NTP epoch and the POSIX epoch
can be simply subtracted from the seconds part of the NTP timestamp.
But the conversion from the fraction part of the long 64-bit NTP timestamp to nanoseconds,
used in the local timestamp structure,
is one of the most problematic tasks for memory constrained systems.
An accurate conversion requires either floating point operations or operations including 64-bit numbers~\cite{c99}.
The conversion is given by
$nsec = fractionl \times 10^9 \div 2^{32}$, where $nsec$ is the nanoseconds part of the local timestamp
and $fractionl$ is the fraction part of the long 64-bit NTP timestamp.

Since there is no native hardware support for floating point nor 64-bit arithmetic operations,
the compiler supplies these operations in form of library, called {\it{libgcc}} in case of the GCC compiler,
which causes a significantly bigger resulting binary file
(kilobytes in case of floating point operations and hundreds of bytes in case of 64-bit operations).
The greatest common divisor of $10^9$ and $2^{32}$ is $2^9$,
so in fact, a relatively simple multiplication of $fractionl$ by $\frac{5^9}{2^{23}}$ must be computed.
This could be computed on 32 bits using sequential
divisions by the power of 2 and multiplications by the power of 5.
In the standard C programming language, the bitwise right shift operator divides the unsigned data type by the power of 2
and the bitwise left shift operator multiplies the unsigned data type by the power of 2~\cite{c99}.
Therefore, the multiplication by 5 can be done using two left shifts and
adding the original value ($5x = 4x + x$).
The only constraint is that the overall coefficient of these operations must not be greater than 1,
that is, the value must be in the range from $0$ to $2^{32}-1$ in every step.
Otherwise, the value could overflow and the result would be incorrect.
Division can not cause such a situation but multiplication could.
The original value could be divided by a greater divisor,
but this would lead to a greater inaccuracy due to the lost of the least significant bits.
Because of this, multiplication done as soon as possible provides more accurate results.
The ideal conversion sequence is therefore given by formula~\ref{equ:conversion}.
\begin{equation}
\label{equ:conversion}
%\frac{10^9}{2^{32}} = \frac{2^9 \times 5^9}{2^9 \times 2^{23}} =
%\frac{5^9}{2^{23}} =
nsec = fractionl \times \frac{5}{2^3} \times \frac{5}{2^3} \times \frac{5}{2^3} \times \frac{5^2}{2^3} \times \frac{5}{2^3}  \times \frac{5}{2^3} \times \frac{5^2}{2^3} \times \frac{1}{2^2}
\end{equation}
It must be noted, that the above presented conversion is not exactly accurate,
because the least significant bits are lost by right shifts. %!LYDIA by
The accuracy can be determined by looping through all the possible values of $fractionl$ %!LYDIA by
and comparing the results with the reference algorithm that uses the floating point operations.
Such a measurement reports the maximum error of 5 nanoseconds,
which is totally adequate for most platforms without the floating point unit or
for platforms where 64-bit multiplication is expensive.
The implementation of the above as well as the program used for the
error measurement can be found on the CD attached to this thesis.
The table of CD contents is listed in appendix~\ref{app:cd-contents}.


After the timestamps were converted, the local clock offset is computed
as given in section~\ref{sec:ntp-algorithms}.
Depending on the absolute value of the local clock offset,
the system time is either set or adjusted using the {\it{clock\_set\_time}}
or {\it{clock\_adjust\_time}} call, respectively.
The clock is set if the time difference is equal or greater than
the offset threshold value.
The NTP specification suggests 0.125~seconds as the default~\cite{rfc5905}.
Because the designed call for setting the time, described in section~\ref{sec:design-interface},
can set the time only within a resolution of one second,
the threshold value must be at least one second.

