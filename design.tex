%=========================================================================
% (c) 2011, 2012 Josef Lusticky

\chapter{Design and analysis}
For implementation of reasonably useful NTP client,
operating system must be able to set, get
and eventually adjust the system time in a similar fashion as
described in section~\ref{sec:system-keeping-and-providing} and~\ref{sec:system-discipline}.
Though not mandatory, adjusting time is important function,
if the time shall be always a monotonically increasing function.
Apart from that, communicating abilities over UDP are also required.

For developing and testing Contiki NTP client,
the AVR Raven platform with 8-bit ATmega1284P CPU~\cite{avr-datasheet} was used.
This platform features IEEE~802.15.4 (Low-Rate Wireless Personal Area Networks) link layer support.
Together with an adaptation layer called 6LoWPAN (IPv6 over Low power Wireless Personal Area Networks)
AVR Raven is able to communicate over IPv6.

How to get a working setup with Contiki on this platform is described in
the document files on the CD enclosed to this thesis.
The CD content hierarchy is listed in appendix~\ref{app:cd-content}.

%Older x86 processors used an interrupt mechanism to switch from
%user-space to kernel-space, but new x86 processors provide instructions
%that optimise this transition (using sysenter and sysexit instructions)~\cite{ibm-linux-system-calls}.
%... This is not in Contiki, as operating systems targeted at embbeded systems produce only 1 binary file. CITATION

%=========================================================================
% (c) 2011, 2012 Josef Lusticky

\section{Network communication}
Thanks to uIP, described in section~\ref{sec:contiki-uip},
the network communication is not a matter for Contiki OS.
% 1 - see implementation/communication.tex
A problem might be a possible packet loss when communication uses UDP on transport layer.
The reason why this can happen often is explained in section~\ref{sec:contiki-uip}.
% 2
In NTP unicast mode, the packet loss might occur either for client's query to server
or for server's response to client.
If the client's query loss occurs, no server response will be sent.
Similarly, if the server's response loss occurs, no message will be received by the client.
Not to block a whole system till the response arrives
is therefore a desired behaviour of the client.

--- to design.tex ?
The AVR Raven platform features 802.15.4 support.
Contiki is used in conjunction with RZ USB Stick - 6LoWPAN.

The 6LoWPAN Adaptation Layer % see Interconnecting smart objects with IP
% IEEE 802.15.4
as RFC~4944 written by 6lowpan working group of IETF
made the underlaying IEEE 802.15.4 layer
look like an IPv6 link~\cite{6lowpan} and
\begin{figure}
  \centering
  \includegraphics[width=9cm,keepaspectratio]{fig/6lowpan.pdf}
  \caption{Communication stack with 6lowpan layer}
  \label{fig:design-6lowpan}
  \bigskip
\end{figure}


Contiki supports broadcast packets as well as sending multicast packets.
Joining multicast groups through Internet Group Management Protocol (IGMP)
and receiving non-local multicast packets
was not supported at the time of writing~\cite{contiki-docs}.
Contiki is also able to use Domain Name System for IPv4 address resolution.
DNS resolution of IPv6 addresses was not implemented in Contiki OS
at the time of writing~\cite{contiki-docs}.


%%=========================================================================
% (c) 2011, 2012 Josef Lusticky

\section{Operating system time interface - TODO}
The main problem of NTP client implementation for Contiki is therefore a total
lack of real-time support.
Not only no common interface is available, but also
almost no platform-specific code has been implemented towards time interface yet.



-- git

Setting the current time is only possible within one second precision -
finer time setting must be made through time adjustments described further.
Implemented {\it{clock\_set\_time}} function computes when the system started
in seconds since the Epoch and saves the result in newly implemented {\it{boottime}} variable.



Unlike GIT!
This variable, counting uptime in seconds,
is particularly used by stimers in Contiki
and modifying it would lead to misbehaviour of stimer library
described in section~\ref{sec:contiki-timers}.


If the operating system implements the kernel discipline described in section~\ref{sec:system-discipline},
an NTP client will announce insertion and deletion of a leap second to the kernel and
the kernel will handle the leap second without further action necessary~\cite{ntp-faq}.
If the operating system does not implement the kernel discipline,
the system clock will show an error of one second after the leap second occurred~\cite{ntp-faq}.
The situation will be handled just like an unexpected change of time -
the operating system will continue with the wrong time for some time,
but an NTP client will step or adjust the time~\cite{ntp-faq}.
This will effectively cause the leap second correction to be applied too late.
% which is a trade-off for smaller memory requirements

Since there is no way of setting, getting and adjusting the time in Contiki OS,
a new interface for setting, getting and eventually
adjusting the time must be developed.


