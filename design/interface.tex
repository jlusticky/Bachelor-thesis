%=========================================================================
% (c) 2011, 2012 Josef Lusticky

\section{Time interface extension}\label{sec:design-interface}
Section~\ref{sec:analysis-time} described, that there is no proper
way of setting, getting and adjusting the time for an NTP client in Contiki OS.
A new interface for setting, getting and eventually adjusting the time
must therefore be developed.

Setting the time should not cause a misbehaviour of the Contiki timers,
described in section~\ref{sec:analysis-time}.
A modification of the {\it{scount}} and the {\it{seconds}} variable must be therefore avoided.
This can be achieved using an additional variable, containing the system boot time,
and modifying only this variable by a call for setting the time.
This way, the {\it{seconds}} variable will be further representing the system uptime
and the current real time can be obtained by $boottime + seconds$.
Since the {\it{scount}} also can not be changed, setting the time is only possible
within a resolution of one second.
The call for adjusting the time takes one parameter - the time to set, that is, the current real time in
seconds since the POSIX prime epoch (1 January 1970).
Because the current real time can not be negative, the parameter is of the unsigned data type.
Listing~\ref{lst:design-settime} shows the call interface for setting the time.
\begin{lstlisting}[caption={Call interface for setting the time},label={lst:design-settime}]
void clock_set_time(unsigned long sec);
\end{lstlisting}

By contrast, a call for getting the current time must be able to provide a higher precision.
The one second resolution is also not adequate for adjusting the time.
Therefore, a new time specification structure must be introduced.
To conform to the POSIX standard~\cite{posix}, this structure consists of two parts.
The structure definition is shown in listing~\ref{lst:design-timespec-structure}.
The structure name was chosen {\it{time\_spec}} to avoid collisions with existing POSIX-compliant systems.
This structure consists of two signed long values representing seconds and nanoseconds.
The nanosecond precision was chosen because modern systems also aim towards this precision
and the microsecond precision would require at least the same data width~\cite{posix,ntp-precision}.
The value 0 seconds and 0 nanoseconds is equal to the POSIX prime epoch (1 January 1970).
In case of the seconds part, the long data type was chosen
because the already present value {\it{seconds}} is also of the long type in Contiki.
To conveniently represent real-time values as well as local clock adjustment values, which may also be negative,
the type was chosen to be signed.
In case of the nanoseconds part, the signed long data type was chosen
to conform to the POSIX standard~\cite{posix}
and to be able to represent positive as well as negative values for local clock adjustments.
\begin{lstlisting}[caption={Time specification structure},label={lst:design-timespec-structure}]
struct time_spec {
  long sec;
  long nsec;
};
\end{lstlisting}

Listing~\ref{lst:design-gettime} shows the call interface for getting the time.
The call for getting the time fills the time specification structure
pointed to by the {\it{ts}} parameter.
The part representing seconds is simply filled with the value of $boottime + seconds$,
while the part representing nanoseconds should be filled with the maximum precision
the clock model allows.
As described in section~\ref{sec:analysis-time},
this can be achieved by reading the {\it{scount}} variable
and by querying the hardware counter that is used for
interrupt generation and includes the time passed since
the last update of the {\it{scount}} variable.
This way, a resolution of
$\frac{1}{CLOCK\_SECOND~\times~counts} = \frac{1}{128~\times~32} = 0.000244140625$~seconds
can be acquired,
where $counts$ is the number of counter register increments between two successive interrupts,
which is 32 by default on AVR Raven, as explained in section~\ref{sec:analysis-clock-interface}.
%If this part was filled by only using the {\it{scount}} variable,
%just the resolution of the timer interrupt frequency would be provided.
\begin{lstlisting}[caption={Call interface for getting the time},label={lst:design-gettime}]
void clock_get_time(struct time_spec *ts);
\end{lstlisting}

Since setting the current time is possible only within one second precision,
finer time setting must be made using the time adjustments.
Section~\ref{sec:analysis-time} explained, that adjusting the time
should use the hardware clock as much as possible.
Therefore, adjusting the time changes the value in {\it{OCR2A}} compare register
to delay or shorten the clock tick interval,
which in turn speeds up or slows down the system time.
To comply with other operating systems,
the amount of required adjustments is specified by the time specification structure.
If the amount of required adjustments is positive, then the system clock is speeded up
until the adjustment has been completed.
If the amount of required adjustments is negative, then the clock is slowed down in a similar fashion.
Because the application specifies the amount of adjustments by the time specification structure,
a new call must be introduced to determinate how many
clock ticks will be delayed or shortened, respectively.
Listing~\ref{lst:design-adjtime} shows the call interface for adjusting the time.
\begin{lstlisting}[caption={Call interface for adjusting the time},label={lst:design-adjtime}]
void clock_adjust_time(struct time_spec *delta);
\end{lstlisting}

As a result of this design, a leap second occurrence will be handled like an unexpected change of time -
the operating system will continue with the wrong system time for some time,
but the NTP client will set or adjust the system time.
This will effectively cause the leap second correction to be applied too late~\cite{ntp-faq},
which is a trade-off for smaller memory requirements.
