%=========================================================================
% (c) 2011, 2012 Josef Lusticky

\section{Network communication}
Thanks to uIP, described in section~\ref{sec:contiki-uip},
the network communication is not a matter for Contiki OS.
% 1 - see implementation/communication.tex
A problem might be a possible packet loss when communication uses UDP on transport layer.
The reason why this can happen often is explained in section~\ref{sec:contiki-uip}.
% 2
In NTP unicast mode, the packet loss might occur either for client's query to server
or for server's response to client.
If the client's query loss occurs, no server response will be sent.
Similarly, if the server's response loss occurs, no message will be received by the client.
Not to block a whole system till the response arrives
is therefore a desired behaviour of the client.



---
LoWPAN is Low-power Wireless Personal Area
Networks (LoWPANs) composed of devices conforming to the IEEE 802.15.4-2003 standard defined
by the IEEE [129]. IEEE 802.15.4 devices are characterized by short range, low bit rate, low power,
and low cost~\cite{interconnecting}.

This can be used conjunction with RZ~USB Stick for communication with desktop PC.
This USB Stick can be loaded with firmware automatically translating
Ethernet packets to 802.15.4 packets.
This firmware is distributed together with Contiki and
is to be found in {\it{/platform/avr-ravenusb/}} directory.

This layer is needed because IPv6 requires support of much larger packet
sizes than the largest IEEE 802.15.4 frame size
(MTU on 802.15.4 links is 127 bytes,
whereas IPv6 mandates links with MTU of at least 1280 bytes)~\cite{interconnecting}.


the 6LoWPAN adaptation layer makes the underlaying IEEE 802.15.4 layer look like an IPv6 link.
\begin{figure}
  \centering
  \includegraphics[width=9cm,keepaspectratio]{fig/6lowpan.pdf}
  \caption{Communication stack with 6lowpan layer}
  \label{fig:design-6lowpan}
  \bigskip
\end{figure}


Contiki supports broadcast packets as well as sending multicast packets.
Joining multicast groups through Internet Group Management Protocol (IGMP)
and receiving non-local multicast packets
was not supported at the time of writing~\cite{contiki-docs}.
Contiki is also able to use Domain Name System for IPv4 address resolution.
DNS resolution of IPv6 addresses was not implemented in Contiki OS
at the time of writing~\cite{contiki-docs}.
