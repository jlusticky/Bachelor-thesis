%=========================================================================
% (c) 2011, 2012 Josef Lusticky

\section{Contiki clock interface}\label{sec:analysis-clock-interface}
As described in section~\ref{sec:contiki-timers},
Contiki provides a basic clock interface particularly for use of timers
with a simple goal - measuring time.
This interface is common for all supported hardware platforms,
but the particular implementation is platform-specific.
The definition of the common interface is located in the {\it{core/sys/clock.h}} file
and the specific implementations can be found in the {\it{clock.c}} file
in the {\it{cpu/}} directory of the Contiki source code.

The clock interface provides the {\it{clock\_init}} call for initialising the hardware clock,
that is automatically called during the boot sequence of Contiki.
On AVR Raven, the {\it{clock\_init}} call sets up
appropriate registers and the interrupt service routine as described in section~\ref{sec:analysis-hwclock}.

This call is implemented as a C macro and is defined in the {\it{cpu/avr/dev/clock-avr.h}} file for AVR CPUs
This macro evaluates to a specific setup code for each different AVR CPU during the compile time.
The setup code is not common to all AVR CPUs because of differences among them - e.g. there are usually
only three Timer/Counter modules, but AVR ATmega1284P has four Timer/Counter modules~\cite{avr-datasheet}.

The hardware clock interrupt, described in section~\ref{sec:analysis-hwclock},
is called {\it{clock tick}}, or {\it{timer tick}}.
At each clock tick, the interrupt service routine increments
a system clock value stored in the memory.
On AVR CPUs, there is a variable counting these clock ticks, called {\it{scount}},
and a variable counting seconds, called {\it{seconds}}.
These variables are defined together with the interrupt service routine in the {\it{cpu/avr/dev/clock.c}} file.
As described in section~\ref{sec:contiki-timers}, there are exactly {\it{CLOCK\_SECOND}} ticks in one second.
When the {\it{scount}} variable reaches the {\it{CLOCK\_SECOND}} value,
the {\it{seconds}} variable is incremented and the {\it{scount}} variable is reset.
The {\it{seconds}} variable is used by the Contiki stimers discussed in section~\ref{sec:contiki-timers}.

To obtain {\it{CLOCK\_SECOND}} interrupts per second, there must be
${\frac{f_{T2}}{CLOCK\_SECOND}}$ counter register increments between two successive interrupts.
On compare match in the CTC mode, the {\it{TCNT2}} counter register is reset to zero as
shown in figure~\ref{fig:design-timing-diagram}.
The value zero is also included in the counting - the 0th count of the timer also takes one tick.
Therefore the value of the {\it{OCR2A}} compare register must be ${\frac{f_{T2}}{CLOCK\_SECOND}} - 1$
in the CTC mode.
The default value of {\it{CLOCK\_SECOND}} for AVR Raven in Contiki is 128,
which implies the default value of the compare register ${\frac{4096}{128}} - 1 = 31$.
The {\it{CLOCK\_SECOND}} value is defined in the {\it{platform/avr-raven/contiki-conf.h}} file.
The value of the compare register is specified in the {\it{clock\_init}} call
and is computed during the compile time.
