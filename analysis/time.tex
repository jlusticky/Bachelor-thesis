%=========================================================================
% (c) 2011, 2012 Josef Lusticky

\section{Time interface}\label{sec:analysis-time}
The low-level clock interface described in previous section
is used by Contiki to provide the system time through the time interface.
Since the value of the {\it{seconds}} variable is zero after the system booted,
it actually expresses the system uptime.
The value of the {\it{seconds}} variable can be obtained by the application using the {\it{clock\_seconds}} call.
However, there is no call for setting this variable in Contiki 2.5.
In the current Git version at the time of writing, a new call {\it{clock\_set\_seconds}}
can be used for this purpose.
Because this call simply rewrites the {\it{seconds}} variable, it breaks the stimer library,
and should be therefore avoided by the NTP client.
Similarly, setting the {\it{scount}} variable would cause
unbalanced increments of the {\it{seconds}} variable.

The precision of one second is also not adequate for the NTP client.
Further precision can be acquired by reading the {\it{scount}} variable,
as it provides a resolution of $\frac{1}{CLOCK\_SECOND}$~seconds.
Moreover, the hardware counter can be also queried, as it includes the time passed since
the last update of the {\it{scount}} variable.
If stimers should not be broken by setting the {\it{seconds}} or {\it{scount}} variable,
and Contiki should be able to set and provide the current time in a higher precision,
a new call interface must be developed.
These calls shall use a structure representing the system time similar
to the existing POSIX standard~\cite{posix}.
Such a structure for representing the time values is also not present in Contiki.

Similarly, there is no call for adjusting the time in Contiki.
Due to memory constraints, software structures controlling the time adjustments are too heavyweight
for use in an embedded operating system running on 8-bit CPUs.
Due to low CPU frequencies, the code of an interrupt service routine can not be complex
and sophisticated clock discipline algorithms should be avoided.
A call for adjusting the time should therefore use abilities
provided by the hardware clock as much as possible.

Updating the value in the {\it{OCR2A}} compare register
can be used for adjusting the time, because decrementing the compare register
value causes a faster increment of the {\it{scount}} variable, which in turn causes
a faster increment of the {\it{seconds}} variable and vice versa.
Such changes would influence the system time and the dependent Contiki timers.
However, applications requiring uninfluenced timers
could use the Contiki rtimers, described in section~\ref{sec:contiki-timers},
because they use a separate hardware clock unaffected by NTP client
(Timer/Counter~3 on AVR Raven platform).
