%=========================================================================
% (c) 2011, 2012 Josef Lusticky <xlusti00@stud.fit.vutbr.cz>

\chapter{NTP client in Contiki OS}
For implementation of a reasonably useful NTP client
operating system must be able to set, get
and eventually adjust time.
Apart from that, ability of communicating over UDP is also required.

Thanks to uIP, described in section~\ref{sec:contiki-uip}, communication is
not a matter for Contiki OS.
Contiki is also able to use Domain Name System for IPv4 address resolution.
DNS resolution of IPv6 addresses was not implemented in Contiki OS
at the time of writing~\cite{contiki-docs}.
The main problem of NTP client implementation for Contiki is therefore total
lack of real-time support.
Not only no common interface availability but also
almost no platform-specific code has been implemented towards time support yet.
Contiki provides basic clock interface particularly for use of timers though.
This interface is common for all supported platforms,
but the particular implementation is platform specific.

Another deal is possible packet loss if communication uses UDP on transport layer.
The reason this can happen often is explained in section~\ref{sec:contiki-uip}.

For developing and testing Contiki NTP client the AVR Raven platform with ATmega1284P~\cite{avr-datasheet} was used.
% do not mention, appendix?
How to get a working setup with Contiki on this platform is described in
%!%% CHECK THIS %%%
docs/setup.pdf file on the CD enclosed to this thesis.
%%% CHECK THIS %%%

\section{Hardware clock interface}
Contiki features a basic clock interface with a simple goal - measuring time.
This interface provides needed calls for timers and its definition is to be found in {\it{/core/sys/clock.h}} file.
Specific implementations of this common interface are located in {\it{/cpu}} directory of Contiki source code.
The interface provides call for initialising CPU's clock system - {\it{clock\_init}}, that is automatically called during
boot sequence of Contiki.
The goal of the {\it{clock\_init}} call is to set up
appropriate counter registers and interrupt service routines. %! TODO .. as described in chapter..
On AVR CPU this call is implemented as C macro which evaluates to specific setup code for each CPU
when compiling.
The setup code is not common to all CPUs because of differences among them - e.g. there are usually
only three Timer/Counter modules, but AVR ATmega1284P has four Timer/Counter modules~\cite{avr-datasheet}.

%At least one of those is always 16 bit wide
%This extra module on AVR ATmega1284P is used for
% three vs. 3

%clock\_seconds
%CLOCK\_SECOND
This is however enough for implementing a reasonable time interface and using it for NTP client later.

% ntp interface extending the clock library, similar to posix calls

%!!AVR

Adjusting time - COMPARE\_REGISTER = 31 => 128Hz => 1s = 1s
FREQ = 32768/8 / 32
COMPARE\_REGISTER = 30 => ca132.129Hz => 1s = ca1.032258s
FREQ = 32768/8 / 31
COMPARE\_REGISTER = 32 => 124.12per => 1s = 0.96p
FREQ = 32768/8 / 33

=> fastest adjust is 0,03s / s


Each TCNT2 increment is $\frac{1}{128 \times 32} \doteq 0,000244$ s
2,44ms minimum clock slew
This is also minimal possible clock adjustment.


Adjustments will influence the timers.
Applications requiring uninfluenced timers
are therefore advised to use rtimers, described in section~\ref{sec:contiki-timers},
because they use separate hardware clock
unaffected by NTP client.

%=========================================================================
% (c) 2011, 2012 Josef Lusticky


Getting the correct current time is only possible if it was set using
the {\it{clock\_set\_time}} function before.
Newly implemented function {\it{clock\_get\_time}} is then able to tell the
current time in seconds since the Epoch by simply adding {\it{boottime}},
and {\it{seconds}}.
%! TODO
Nanoseconds part is filled using {\it{scount}} variable counting number of
interrupts within a second.
Since this variable is incremented every interrupt and there are {\it{CLOCK\_SECOND}} interrupts
per second, it is possible to get resolution of $\frac{1~000~000~000}{CLOCK\_SECOND}$ ns.
The same resolution have Contiki timers, described in section~\ref{sec:contiki-timers}.
\begin{lstlisting}
void
clock_get_time(struct time_spec *ts)
{
  ts->sec = boottime + seconds;
  ts->nsec = (1000000000 / CLOCK_SECOND) * scount;
}
\end{lstlisting}
Because {\it{1000000000}} and {\it{CLOCK\_SECOND}} are both constants, the compiler is able to
calculate the result of division during compile time.
Furthermore as both numbers are integers, the result is integer as well~\cite{c99}.
The most of CPU time is therefore spent on multiplication.
E.g. if the code is compiled using GCC version 4.3.5,
multiplication of two 32-bit variables takes 33 instructions including {\it{call}} and {\it{ret}}
instructions for entering and returning from the {\it{\_\_mulsi3}} routine, which computes
the result of multiplication.
%avr-objdump
According to AVR Instruction Set manual~\cite{avr-instruction-set},
this results in 48 clock cycles overhead,
which takes 3~000 nanoseconds assuming 16MHz CPU clock.
The timestamp provided is therefore not exact.
However, since this consumed time strongly depends on architecture and compiler specifications,
no correction was implemented to remove this inaccuracy.
The application must be instead aware that the timestamp is not exactly accurate.

TODO: Greater precision is further implemented by reading counter register.

TODO: Adjust time
POSIX:
Time values that are between two consecutive non-negative integer multiples
of the resolution of the specified clock are truncated down to the smaller multiple of the resolution.


%=========================================================================
% (c) 2011, 2012 Josef Lusticky <xlusti00@stud.fit.vutbr.cz>

\section{NTP client implementation}
Structures representing NTP message were borrowed from OpenNTPD NTP Unix daemon.
%They are not using the GCC extension for representing a bit field.

Packet sanity tests~\cite{ntp-arch}.

A client sends messages to each server with a poll interval of $2^{\tau}$
seconds, as determined by the poll exponent $\tau$ (tau).
In NTPv4, $\tau$ ranges from 4 (16 s) to 17 (36 h).


% filling the packet
Precision express strictly speaking elapsed time to read the system clock from userland~\cite{ntp-arch}.
However most implementation supply clock precision.
%Dragonfly BSD:
%wmsg.precision = -6;
%Chrony, NTP.org - getting resolution, gettimeofday, clock_getres
\begin{lstlisting}
// set clock precision - convert Hz to log2 - borrowed from OpenNTPD
int b = CLOCK_SECOND; // CLOCK_SECOND * OCR2A
int a;
for (a = 0; b > 1; a--, b >>= 1)
  {}
msg.precision = a;
\end{lstlisting}
This will work for clock precision greater or equal 1s, i.e. CLOCK\_SECOND must be greater or equal 1.
%refer to CLOCK\_SECOND - is always greater or equal 1

The clock are set if the time difference is greater than XX seconds. %! TODO
\begin{lstlisting}
if (labs((signed long) (ts.sec - tmpts.sec)) > 2)
{
  clock_set_time(ts.sec);
}
\end{lstlisting}
Even if {\it{tmpts.sec}} value is greater than {\it{ts.sec}} value,
subtracting and casting to signed type gives correct (negative) result~\cite{c99}.
Assuming 32-bit data types this will work until 2038 when wrap around can occur due to difference
between {\it{ts.sec}} and {\it{tmpts.sec}} greater than $2^{31}$-1 (2~147~483~647).
But as NTP Era 0 ends 2036 the NTP client code must be changed in the future anyway.

%! TODO

%Adjusting time
%1/128/32 = 0.000244141
%0.000244141x32x127+0.000244141x31 == smallest possible adjustment == 244us

%\section{NTP values and convertions}
Unlike the RFC 5905~\cite{rfc5905} shows, there are no 64 bit values. %! RFC - A.4. Kernel System Clock Interface
No floating point numbers - library.
Division of unsigned integer number by 2 is automatically translated by compiler to logical right shift,
making it fast operation.

Converting between NTP and local timestamps requires floating point operations or 64-bit numbers.
According to output from avr-size tool, using 64-int number for conversion
uses 4~330 bytes more in program section of resulted binary file 
and floating point operation takes 3~474 bytes more
than algorithm developed?

C99 - shift E1 >> E2: if E1 has a signed type and a nonnegative value, the value of
the result is the integral part of the quotient of $E1 / 2^{E2}$.

In case of floating point operations, the libgcc is used.
\url{http://gcc.gnu.org/onlinedocs/gccint/Libgcc.html}

%
% NEGATIVE result for the first time
In some scenarios where the initial frequency offset of the client is
  relatively large and the actual propagation time small, it is
   possible for the delay computation to become negative.  For instance,
   if the frequency difference is 100 ppm and the interval T4-T1 is 64
   s, the apparent delay is -6.4 ms.  Since negative values are
   misleading in subsequent computations, the value of delta should be
   clamped not less than s.rho, where s.rho is the system precision
   described in Section 11.1, expressed in seconds~\cite{rfc5905}.
%


%=========================================================================
% (c) 2011, 2012 Josef Lusticky

\section{Network communication}
The 6LoWPAN Adaptation Layer % see Interconnecting smart objects with IP
% IEEE 802.15.4
as RFC~4944 written by 6lowpan working group of IETF
made the underlaying IEEE 802.15.4 layer
look like an IPv6 link~\cite{6lowpan} and
\begin{figure}
  \centering
  \includegraphics[width=9cm,keepaspectratio]{fig/6lowpan.pdf}
  \caption{Communication stack with 6lowpan layer}
  \label{fig:implementation-6lowpan}
  \bigskip
\end{figure}

NTP server can be specified in Makefile or
during compilate time using {\it{REMOTE\_HOST}} define.
%! TODO
TODO: If no host is specified,
NTP client assumes NTP broadcast communication mode.


% There is no IPv4 support...

%The uIP packet buffer is accessed through
%the uip\_buf array and is used to hold incoming and outgoing packets.
%The device driver should place incoming data into this buffer.
%When sending data, the device driver should read the link
%level headers and the TCP/IP headers from this buffer.
%The size of the link level headers is configured by the UIP\_LLH\_LEN
%define and in case of ethernet it is 14.

%The application data need not be placed in this buffer, so
%the device driver must read it from the place pointed to by the
%uip\_appdata pointer %as illustrated by the following example:

Routing to Meinberg NTP primary server.
Measurements made using this setup are discussed in chapter~\ref{chap:measurements}.

