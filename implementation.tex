%=========================================================================
% (c) 2011, 2012 Josef Lusticky <xlusti00@stud.fit.vutbr.cz>

\chapter{NTP client in Contiki OS}
%\!Distinguish between TIME and CLOCK

For implementation of a reasonably useful NTP client
operating system must meet conditions listed in appendix~\ref{app:requirements}.
The main problem for NTP client implementation in Contiki is a total
lack of time support.
Not only no common interface availability but also
almost no platform-specific code has been implemented towards time support yet.
Contiki provides basic clock interface particularly for use of timers though.
This interface is is common to for all supported platforms, but the particular implementation
is platform specific.

Another deal is possible packet loss if communication uses UDP on transport layer.
The reason while this can often happen is explained in section~\ref{sec:contiki-uip}.

For developing and testing Contiki NTP client the AVR Raven platform was used.
%How to get a working set up with Contiki on this platform is described in documents on the CD enclosed to this thesis.

\section{Clock interface}
Contiki feature a basic clock interface with a simple goal - measuring time.
This interface provides needed calls for timers and its defintion is to be found in /core/sys/clock.h file.
Specific implementations of this common interface are localed in /cpu direcory of Contiki source code.
Interface provides call for initialising CPU's clock system - clock\_init, that is automatically called during
boot sequence of Contiki.
The goal of the clock\_init call is to set up
appropriate timer/counter registers and interrupt service routines.
%On AVR CPUs this call is implemented as C macro ...
%clock\_seconds
%CLOCK\_SECOND
This is however enough for implementing a reasonable time interface and later using it for NTP client.

% ntp interface extending the clock library, similiar to posix calls

%\!!AVR


\section{NTP client implementation}
The structures representing NTP message were borrowed from OpenNTPD NTP Unix daemon.
