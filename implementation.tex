%=========================================================================
% (c) 2011, 2012 Josef Lusticky

\chapter{Implementation}
TODO: write introduction..
%... This can be effectively solved by NTP Poll Interval.

\section{Time specification structure}
%Since there is no way of setting, getting and adjusting the time in Contiki OS,
%the interface for setting, getting and adjusting time was developed in this thesis.
New structure for expressing time values was implemented.
This structure is similar to POSIX {\it{timespec}} structure,
as described in section~\ref{sec:analysis-interface}.
However, name was chosen {\it{time\_spec}} to avoid collision with
existing POSIX-compliant systems.
\begin{lstlisting}
struct time_spec {
  long sec;
  long nsec;
};
\end{lstlisting}
This structure consists of two signed long values for expressing seconds and nanoseconds.
The value 0 seconds and 0 nanoseconds is equal to Unix prime epoch (1st January 1970).
In case of seconds part, the 32-bit signed long value was chosen because
it can conveniently
represent real-time values as well as local clock adjustment values, which may also be negative.
%! The existing value {\it{seconds}}, representing uptime in Contiki, is of the unsigned long type.
Such time representation will wrap around in year 2038 and is facing
what is commonly known as the year 2038 problem~\cite{posix}.

In case of nanoseconds part, the 32-bit signed long value was chosen because
one second has $10^9$ nanoseconds and it is
desirable to be able to express positive as well as negative values for local clock adjustments.
As 32-bit signed long type shall be able to express values from -$2^{31}$-1 (-2~147~483~647)
to $2^{31}$-1 (2~147~483~647)~\cite{c99},
such representation will therefore never wrap around.

It should be noted that signed long data type does not have to always result in a 32-bit variable -
it is up to compiler what data width it chooses for each data type.
But as ISO C99 standard states, the maximum value for an object of type signed long
shall be greater or equal $2^{31}$-1 (2~147~483~647)~\cite{c99}.
This in fact results in at least 32-bit variable unless the compilation setting is changed.
Next to this, as the already presented variable {\it{seconds}} is of unsigned long type,
the value {\it{sec}} in {\it{time\_spec}} structure %and {\it{boottime}}
shall be of the same data width as arithmetic operations are made on them.

Usage of unsigned data type delays the wrap around problem to year 2106,
however will disable use of negative values needed for adjusting local clock.
%As current NTP Era ends 2036 code has to be changed anyway...~\cite{ntp-y2k}.


\section{Setting the time} %TODO
Setting the current time is only possible within one second precision -
finer time setting must be made through time adjustments described further.
Implemented {\it{clock\_set\_time}} function computes when the system started
in seconds since the Epoch and saves the result in newly implemented {\it{boottime}} variable.

Not only no variables incremented every interrupt nor any internal clock registers
are affected, but also already presented variable {\it{seconds}} is not modified.
Modifying this variable would lead to misbehaviour of stimer library
described in section~\ref{sec:contiki-timers}.

Thanks to this newly implemented {\it{clock\_set\_time}} function and {\it{boottime}} variable,
running Contiki system is able to tell uptime, current time and
time when was the system booted.
\begin{lstlisting}
volatile unsigned long boottime;

void
clock_set_time(unsigned long sec)
{
  boottime = sec - seconds;
}
\end{lstlisting}

%%%

%=========================================================================
% (c) 2011, 2012 Josef Lusticky


Getting the correct current time is only possible if it was set using
the {\it{clock\_set\_time}} function before.
Newly implemented function {\it{clock\_get\_time}} is then able to tell the
current time in seconds since the Epoch by simply adding {\it{boottime}},
and {\it{seconds}}.
%! TODO
Nanoseconds part is filled using {\it{scount}} variable counting number of
interrupts within a second.
Since this variable is incremented every interrupt and there are {\it{CLOCK\_SECOND}} interrupts
per second, it is possible to get resolution of $\frac{1~000~000~000}{CLOCK\_SECOND}$ ns.
The same resolution have Contiki timers, described in section~\ref{sec:contiki-timers}.
\begin{lstlisting}
void
clock_get_time(struct time_spec *ts)
{
  ts->sec = boottime + seconds;
  ts->nsec = (1000000000 / CLOCK_SECOND) * scount;
}
\end{lstlisting}
Because {\it{1000000000}} and {\it{CLOCK\_SECOND}} are both constants, the compiler is able to
calculate the result of division during compile time.
Furthermore as both numbers are integers, the result is integer as well~\cite{c99}.
The most of CPU time is therefore spent on multiplication.
E.g. if the code is compiled using GCC version 4.3.5,
multiplication of two 32-bit variables takes 33 instructions including {\it{call}} and {\it{ret}}
instructions for entering and returning from the {\it{\_\_mulsi3}} routine, which computes
the result of multiplication.
%avr-objdump
According to AVR Instruction Set manual~\cite{avr-instruction-set},
this results in 48 clock cycles overhead,
which takes 3~000 nanoseconds assuming 16MHz CPU clock.
The timestamp provided is therefore not exact.
However, since this consumed time strongly depends on architecture and compiler specifications,
no correction was implemented to remove this inaccuracy.
The application must be instead aware that the timestamp is not exactly accurate.

TODO: Greater precision is further implemented by reading counter register.

TODO: Adjust time
POSIX:
Time values that are between two consecutive non-negative integer multiples
of the resolution of the specified clock are truncated down to the smaller multiple of the resolution.


%=========================================================================
% (c) 2011, 2012 Josef Lusticky <xlusti00@stud.fit.vutbr.cz>

\section{NTP client implementation}
Structures representing NTP message were borrowed from OpenNTPD NTP Unix daemon.
%They are not using the GCC extension for representing a bit field.

Packet sanity tests~\cite{ntp-arch}.

A client sends messages to each server with a poll interval of $2^{\tau}$
seconds, as determined by the poll exponent $\tau$ (tau).
In NTPv4, $\tau$ ranges from 4 (16 s) to 17 (36 h).


% filling the packet
Precision express strictly speaking elapsed time to read the system clock from userland~\cite{ntp-arch}.
However most implementation supply clock precision.
%Dragonfly BSD:
%wmsg.precision = -6;
%Chrony, NTP.org - getting resolution, gettimeofday, clock_getres
\begin{lstlisting}
// set clock precision - convert Hz to log2 - borrowed from OpenNTPD
int b = CLOCK_SECOND; // CLOCK_SECOND * OCR2A
int a;
for (a = 0; b > 1; a--, b >>= 1)
  {}
msg.precision = a;
\end{lstlisting}
This will work for clock precision greater or equal 1s, i.e. CLOCK\_SECOND must be greater or equal 1.
%refer to CLOCK\_SECOND - is always greater or equal 1

The clock are set if the time difference is greater than XX seconds. %! TODO
\begin{lstlisting}
if (labs((signed long) (ts.sec - tmpts.sec)) > 2)
{
  clock_set_time(ts.sec);
}
\end{lstlisting}
Even if {\it{tmpts.sec}} value is greater than {\it{ts.sec}} value,
subtracting and casting to signed type gives correct (negative) result~\cite{c99}.
Assuming 32-bit data types this will work until 2038 when wrap around can occur due to difference
between {\it{ts.sec}} and {\it{tmpts.sec}} greater than $2^{31}$-1 (2~147~483~647).
But as NTP Era 0 ends 2036 the NTP client code must be changed in the future anyway.

%! TODO

%Adjusting time
%1/128/32 = 0.000244141
%0.000244141x32x127+0.000244141x31 == smallest possible adjustment == 244us

%\section{NTP values and convertions}
Unlike the RFC 5905~\cite{rfc5905} shows, there are no 64 bit values. %! RFC - A.4. Kernel System Clock Interface
No floating point numbers - library.
Division of unsigned integer number by 2 is automatically translated by compiler to logical right shift,
making it fast operation.

Converting between NTP and local timestamps requires floating point operations or 64-bit numbers.
According to output from avr-size tool, using 64-int number for conversion
uses 4~330 bytes more in program section of resulted binary file 
and floating point operation takes 3~474 bytes more
than algorithm developed?

C99 - shift E1 >> E2: if E1 has a signed type and a nonnegative value, the value of
the result is the integral part of the quotient of $E1 / 2^{E2}$.

In case of floating point operations, the libgcc is used.
\url{http://gcc.gnu.org/onlinedocs/gccint/Libgcc.html}

%
% NEGATIVE result for the first time
In some scenarios where the initial frequency offset of the client is
  relatively large and the actual propagation time small, it is
   possible for the delay computation to become negative.  For instance,
   if the frequency difference is 100 ppm and the interval T4-T1 is 64
   s, the apparent delay is -6.4 ms.  Since negative values are
   misleading in subsequent computations, the value of delta should be
   clamped not less than s.rho, where s.rho is the system precision
   described in Section 11.1, expressed in seconds~\cite{rfc5905}.
%


%=========================================================================
% (c) 2011, 2012 Josef Lusticky

\section{Network communication}
% 1 - see design/network.tex
Communication over IEEE~802.15.4 link layer uses a 6LoWPAN adaptation layer.
Thanks to this layer, AVR Raven running Contiki OS is connected to the IPv6 Internet.
The client and the developed interface uses no IP version specific code,
therefore a communication over IPv4 should be also possible, though not tested.

The packet loss problem was described in section~\ref{sec:design-network}.
However, packet loss is not a matter for NTP if using either broadcast or unicast mode.
In broadcast mode, lost server packet causes no setting or adjusting time of client's
local clock.
The client simply waits without disruption for next NTP broadcast message.
If client needs to figure out it's local clock offset at the moment,
it can simply query a server using NTP unicast mode.
% 2
Upon sending the packet, the NTP client process yields
using the {\it{PROCESS\_YIELD}} statement, so no active waiting
causes blocking the whole system.

The remote NTP server can be specified in Makefile or
using the {\it{REMOTE\_HOST}} define macro.
If no remote host is specified,
NTP client assumes only NTP broadcast communication mode will be used.
The broadcast mode is intended particularly for energy or memory constrained clients
or for a huge number of NTP clients and a single NTP server
in a network with small propagation delay.
Should there be more devices running Contiki present in one network,
each of them needs a different link layer address.
This address can be configured in Makefile as well.
Beware that although the firmware for RZ~USB Stick automatically translates
between Ethernet link layer addresses and IEEE~802.15.4 link layer
addresses, both are are still different links and can not be mixed in one
layer~2 network (e.g. bridging will not work).

Dynamic increasing or decreasing the client's poll interval in response to
Kiss-o'-Death packets, described in section~\ref{sec:ntp-network}, is not implemented.
The configuration instead assumes, that an exhausted NTP server rather drops the incoming
client's packet than sending the response with KoD code.

Contiki NTP Client is primarily intended for use in local networks with a single master NTP server,
although using any NTP server found in the Internet would work.
Figure~\ref{fig:implementation-routing} shows the network topology used
for tests and measurements of Contiki NTP Client.
% How to set up the illustrated network is described in tutorial on CD.
The Meinberg primary NTP server was synchronised with PPS %todo.
Measurements made using this setup are discussed in chapter~\ref{chap:measurements}.

\begin{figure}
	\centering
	\includegraphics[width=10cm,keepaspectratio]{fig/radvd-routing.png}
	\caption{Contiki NTP Client communicating with remote NTP server}
	\label{fig:implementation-routing}
	%\bigskip
\end{figure}

