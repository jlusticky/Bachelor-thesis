%=========================================================================
% (c) 2011, 2012 Josef Lusticky <xlusti00@stud.fit.vutbr.cz>

\chapter{Hardware concerns for implementing real-time support}
%! review
A typical desktop computer today includes CPU based on Intel x86 architecture.
Real-Time Clock (RTC) in CMOS memory that is battery powered

Unfortunately Intel x86 architecture is heavily influenced by backwards compatibility,
e.g. the time value can also be stored in Binary Code Digit (BCD) encoding in RTC.

In year 19xx / Starting with Intel 386
Intel introduced
Programmable Interrupt Controller (PIT) Intel 8253 and 8254 - 3 counters (counter 0 interrupt to OS)


Used by historic versions of Linux
=> read initial time from RTC, setup PIT and interrupts (IRQ 0, INT 8), increment jiffies on every interrupt, provide app resolution of jiffies

init/main.c - time\_init() - read from RTC and save to startup\_time
kernel/sched.c - sched\_init() = PTI setup for interrupts - LATCH (1193180/HZ)
kernel/system\_call.s - timer\_interrupt() in assembly - increments jiffies

The current real time is provided by CURRENT\_TIME (startup\_time+jiffies/HZ) => since jiffies is integer and HZ is 100 => resolution of 10ms.
kernel/sys.c - sys\_time() - CURRENT\_TIME returned


\section{NTP on POSIX-compliant systems}

POSIX standard
\begin{lstlisting}[morekeywords={clockid_t,time_t},numbers=none]
clock_gettime(clockid_t clock_id, struct timespec *res);
clock_settime(clockid_t clock_id, const struct timespec *res);
clock_getres(clockid_t clock_id, struct timespec *res);

struct timespec
time_t  tv_sec    Seconds
long    tv_nsec   Nanoseconds
\end{lstlisting}
A clock may be system-wide (that is, visible to all processes)
or per-process (measuring time that is meaningful only within a process).
All implementations shall support a clock\_id of CLOCK\_REALTIME as
defined in <time.h>.
This clock represents the clock measuring real time for the system.
For this clock, the values returned by clock\_gettime() and specified
by clock\_settime() represent the amount
of time (in seconds and nanoseconds) since the Epoch.


Operating systems
Linux
OpenBSD
DragonflyBSD

NTP implementations
NTP from ntp.org project - reference implementation
Chrony
OpenNTPD - OpenBSD
dntpd - DragonflyBSD
clockspeed
