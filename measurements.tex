%=========================================================================
% (c) 2011, 2012 Josef Lusticky

\chapter{Measurements}\label{chap:measurements}
There are several factors that can be measured.
Clock interrupt frequency measurements show the influence of clock adjustments
on the number of clock ticks (interrupts) per second.
Clock offset measurements show the time difference between the reference clock and
the local clock.
Clock phase measurements show the phase difference between the reference clock and
the local clock, that is, when each second is accounted.

\section{Clock interrupt frequency}
For measuring clock interrupt frequency the bit 7 of Port D
and ground pin was connected to UNI-T~2025CEL digital oscilloscope.
At the beginning of the interrupt service routine a logic 1 was written,
what causes high level of voltage.
At the end of the interrupt service routine a logic 0 was written,
what causes low voltage.

When there is no clock adjustment, the value in output compare register is 31 by default.
The clock interrupt frequency
should be equal to a~value of {\it{CLOCK\_SECOND}} macro, which is 128 by default on AVR~Raven.
The figure~\ref{fig:measurements-osc-no-adjust} is showing the output from oscilloscope
for this case.

When slowing down the clock, the clock interrupt frequency
should be equal to
$$\frac{\frac{f_{asy}}{prescaler}}{inc + 1} = \frac{\frac{32768}{8}}{31+1} = 124.\overline{12}$$
The figure~\ref{fig:measurements-osc-speed-up} is showing the output from oscilloscope
for this case.

When speeding up the clock, the clock interrupt frequency
should be equal to
$$\frac{\frac{f_{asy}}{prescaler}}{inc - 1} = \frac{\frac{32768}{8}}{31-1} \doteq 132.129$$
The figure~\ref{fig:measurements-osc-slow-down} is showing the output from oscilloscope
for this case.

The measured values are not exact equal to those expected.
This is mostly due to influence of the clock source
(32~768~Hz quartz crystal oscillator) by a room temperature,
but it could also be air pressure or magnetic fields, etc.

\section{Clock offset}


\section{Clock phase}
The GPS based clock Meinberg~GPS~167 and digital oscilloscope UNI-T~2025CEL
was used for measuring clock frequency difference.
Meinberg~GPS~167 rises an impulse when each second is accounted.
Contiki on AVR~Raven was configured to write a logic~1
to~bit~7 of~Port~D when each second is accounted,
and to write a logic~0 to~the same bit after~25 clock ticks.

TODO: FIGURE
In phase.
TODO: FIGURE
Out of phase.
