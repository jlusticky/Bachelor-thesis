%=========================================================================
% (c) 2011, 2012 Josef Lusticky

\chapter{Measurements}\label{chap:measurements}
There are several factors that can be measured.
Clock interrupt frequency measurements show the influence of clock adjustments
on the number of clock ticks (interrupts) per second.
Clock offset measurements show the time difference between the reference clock and
the local clock.
Clock phase measurements show the phase difference between the reference clock and
the local clock, that is, when each second is accounted.

\section{Clock interrupt frequency}
For measuring clock interrupt frequency the bit 7 of Port D
and ground pin was connected to UNI-T~2025CEL digital oscilloscope.
At the beginning of the interrupt service routine a~logic 1 is written,
what causes high level of voltage.
At the end of the interrupt service routine a~logic 0 is written,
what causes low voltage.

When there is no clock adjustment, the value in output compare register is 31 by default.
The clock interrupt frequency
is supposed to be equal to a~value of {\it{CLOCK\_SECOND}} macro, which is 128 by default on AVR~Raven.
The~figure~\ref{fig:osc-no-adjust} shows the output from oscilloscope
for this case.
$$\frac{\frac{f_{asy}}{prescaler}}{inc} = \frac{\frac{32768}{8}}{32} = 128$$

The~figure~\ref{fig:osc-speed-up} shows the~output from oscilloscope
when slowing down the clock.
The~clock interrupt frequency
is supposed to be equal to $124.\overline{12}$~Hz.
$$\frac{\frac{f_{asy}}{prescaler}}{inc + 1} = \frac{\frac{32768}{8}}{32+1} = 124.\overline{12}$$

The~figure~\ref{fig:osc-slow-down} shows the output from oscilloscope
when speeding up the clock.
The~clock interrupt frequency
is supposed to be approximately equal to 132.129~Hz.
$$\frac{\frac{f_{asy}}{prescaler}}{inc - 1} = \frac{\frac{32768}{8}}{32-1} \doteq 132.129$$


The~measured values are not exactly equal to those expected.
This is mostly due to influence of the clock source
(32~768~Hz quartz crystal oscillator) by a room temperature,
but it could also be air pressure or magnetic fields, etc.

\section{Clock offset}
For measuring clock offset the serial output from AVR~Raven was used.
When the developed NTP client receives response from the server,
it calculates the local clock offset and prints this value to the serial output.
The NTP poll interval was set to 16 seconds, that means, the local clock offset
is calculated every 16 seconds.

\section{Clock phase}
The GPS based clock Meinberg~GPS~167 and digital oscilloscope UNI-T~2025CEL
was used for measuring clock frequency difference.
Meinberg~GPS~167 rises an impulse when each second is accounted.
Contiki on AVR~Raven was configured to write a logic~1
to~bit~7 of~Port~D when each second is accounted,
and to write a logic~0 to~the same bit after~25 clock ticks.

Figure~\ref{fig:osc-out-of-phase} shows the clock out of phase
after setting the time using the {\it{clock\_set\_time}} call.
When NTP client uses the {\it{clock\_adjust\_time}} call,
the local clock offset as well as the phase is being adjusted.
Figure~\ref{fig:osc-adjusting-phase} shows the phase while adjusting the clock.
Finally, when the clock adjustments finish, the clock is in phase with
the reference clock. Figure~\ref{fig:osc-in-phase} shows the clock in phase after adjustments finished.
